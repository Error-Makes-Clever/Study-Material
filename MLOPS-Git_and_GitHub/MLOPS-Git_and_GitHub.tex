\documentclass[11pt,a4paper]{article}
\usepackage[margin=1in]{geometry}
\usepackage{amsmath}
\usepackage{amssymb}
\usepackage{titlesec}
\usepackage{enumitem}
\usepackage{xcolor}
\usepackage{tcolorbox}
\usepackage{fancyhdr}
\usepackage{listings}
\usepackage{hyperref}

% Header and Footer
\pagestyle{fancy}
\fancyhf{}
\rhead{Git \& GitHub Complete Guide}
\lhead{Version Control Systems}
\cfoot{\thepage}

% Title formatting
\titleformat{\section}{\Large\bfseries\color{blue!70!black}}{\thesection}{1em}{}[\titlerule]
\titleformat{\subsection}{\large\bfseries\color{blue!50!black}}{\thesubsection}{1em}{}

% Code styling
\definecolor{codebg}{gray}{0.95}
\lstset{
    backgroundcolor=\color{codebg},
    basicstyle=\ttfamily\small,
    breaklines=true,
    frame=single,
    numbers=left,
    numberstyle=\tiny,
    xleftmargin=2em,
    framexleftmargin=1.5em
}

% Command box
\newtcolorbox{cmdbox}{
    colback=codebg,
    colframe=black!50,
    boxrule=0.5pt,
    left=2mm,
    right=2mm,
    top=1mm,
    bottom=1mm
}

\begin{document}

% Title Page
\begin{titlepage}
    \centering
    \vspace*{2cm}
    {\Huge\bfseries Git \& GitHub\\[0.5cm] Complete Reference Guide\par}
    \vspace{1cm}
    {\Large Version Control Systems\par}
    \vspace{2cm}
    {\large A Comprehensive Guide to Git Commands,\\
    GitHub Workflows, and Best Practices\par}
    \vspace{3cm}
    {\Large\bfseries Sujil S\par}
    \vspace{0.5cm}
    {\large\texttt{sujil9480@gmail.com}\par}
    \vfill
    {\large \today\par}
\end{titlepage}

\tableofcontents
\newpage

% ========================
% SECTION 1: INTRODUCTION
% ========================
\section{Introduction to Git}

\subsection{What is Git?}
Git is a \textbf{distributed version control system} (DVCS) designed to manage source code efficiently. It enables developers to:
\begin{itemize}[leftmargin=*]
    \item Track changes in code over time
    \item Collaborate effectively with multiple developers
    \item Support flexible and non-linear development workflows
    \item Maintain complete project history locally and remotely
\end{itemize}

\subsection{Key Features of Git}
\begin{enumerate}[leftmargin=*]
    \item \textbf{Version Control}: Complete history of all changes
    \item \textbf{Distributed Nature}: Every developer has a full repository copy
    \item \textbf{Branching \& Merging}: Parallel development support
    \item \textbf{Fast Performance}: Efficient operations on large codebases
    \item \textbf{Data Integrity}: SHA-based commit identification
\end{enumerate}

\subsection{Why Use Git?}

\subsubsection{Bug Fixing and Code Versioning}
\begin{itemize}[leftmargin=*]
    \item Maintains complete history of code changes
    \item Tracks \textit{who} made changes, \textit{when}, and \textit{why}
    \item Enables reverting to previous stable versions
    \item Supports multiple software versions simultaneously
    \item Allows safe testing without affecting main codebase
\end{itemize}

\subsubsection{Collaboration}
\begin{itemize}[leftmargin=*]
    \item Multiple developers work on same project simultaneously
    \item Manages and resolves conflicts between changes
    \item Each developer works on local repository copy
    \item Pull requests and code reviews improve quality
    \item Enhances teamwork in large projects
\end{itemize}

\subsubsection{Non-Linear Development}
\begin{itemize}[leftmargin=*]
    \item Branching supports parallel development
    \item Separate branches for features, bug fixes, experiments
    \item Independent work without disturbing main code
    \item Tested branches merge into main branch
    \item Improves flexibility and development speed
\end{itemize}

\newpage

% ========================
% SECTION 2: CORE CONCEPTS
% ========================
\section{Core Git Concepts}

\subsection{Distributed Version Control System (DVCS)}
\begin{itemize}[leftmargin=*]
    \item Every developer has a complete repository copy
    \item Full project history available locally
    \item Supports offline work
    \item Improves collaboration and redundancy
    \item Example: Git, Mercurial
\end{itemize}

\subsection{Repository}
A \textbf{repository} (or "repo") is a storage location for a Git project containing:
\begin{itemize}[leftmargin=*]
    \item All project files and directories
    \item Complete commit history
    \item All branches and tags
    \item Configuration and metadata
\end{itemize}

Repositories can be:
\begin{itemize}[leftmargin=*]
    \item \textbf{Local}: Stored on developer's machine
    \item \textbf{Remote}: Hosted on servers (GitHub, GitLab, Bitbucket)
\end{itemize}

\subsection{Commit}
A \textbf{commit} is a snapshot of staged changes:
\begin{itemize}[leftmargin=*]
    \item Each commit has a unique SHA ID (hash)
    \item Contains commit message describing changes
    \item Forms the project's version history
    \item Immutable once created
\end{itemize}

\subsection{Branch}
A \textbf{branch} is a parallel line of development:
\begin{itemize}[leftmargin=*]
    \item Allows working on features independently
    \item Main/master branch contains stable code
    \item Feature branches for new development
    \item Can be merged back into main branch
\end{itemize}

\subsection{Working Directory}
The \textbf{working directory} is where files are created and modified:
\begin{itemize}[leftmargin=*]
    \item Contains current project files
    \item Changes not tracked until staged
    \item Reflects current branch state
\end{itemize}

\subsection{Staging Area (Index)}
The \textbf{staging area} is an intermediate preparation zone:
\begin{itemize}[leftmargin=*]
    \item Prepares changes for commit
    \item Allows selective file tracking
    \item Files added using \texttt{git add}
    \item Review changes before committing
\end{itemize}

\subsection{Merge}
\textbf{Merging} combines changes from different branches:
\begin{itemize}[leftmargin=*]
    \item Integrates parallel development work
    \item Creates merge commit (in complex merges)
    \item May require conflict resolution
\end{itemize}

\subsection{Clone}
\textbf{Cloning} creates a local copy of remote repository:
\begin{itemize}[leftmargin=*]
    \item Includes all files, branches, and history
    \item Automatically sets up remote reference
    \item Enables local development
\end{itemize}

\newpage

% ========================
% SECTION 3: GIT WORKFLOW
% ========================
\section{Git Framework and Workflow}

\subsection{The Four-Stage Architecture}

Git follows a structured workflow through four main areas:

\begin{enumerate}[leftmargin=*]
    \item \textbf{Workspace (Working Directory)}
    \begin{itemize}
        \item Where files are created, modified, or deleted
        \item Changes not tracked until staged
        \item Example: Creating/editing \texttt{index.html}
    \end{itemize}
    
    \item \textbf{Staging Area (Index)}
    \begin{itemize}
        \item Intermediate area for preparing commits
        \item Allows selective change tracking
        \item Command: \texttt{git add <file>}
    \end{itemize}
    
    \item \textbf{Local Repository}
    \begin{itemize}
        \item Stores committed project history locally
        \item Each commit is a snapshot of staged changes
        \item Command: \texttt{git commit -m "message"}
    \end{itemize}
    
    \item \textbf{Remote Repository}
    \begin{itemize}
        \item Hosted on servers (GitHub, GitLab)
        \item Enables team collaboration
        \item Command: \texttt{git push origin main}
    \end{itemize}
\end{enumerate}

\subsection{Typical Workflow Example}
\begin{enumerate}[leftmargin=*]
    \item Developer edits \texttt{index.html} in workspace
    \item File added to staging area: \texttt{git add index.html}
    \item Staged changes committed to local repository
    \item Commit pushed to remote repository
\end{enumerate}

\begin{tcolorbox}[colback=yellow!10!white,colframe=orange!75!black,title=Key Point]
The Git framework ensures systematic version control by organizing changes from development to collaboration through distinct stages.
\end{tcolorbox}

\newpage

% ========================
% SECTION 4: INITIAL SETUP
% ========================
\section{Git Configuration and Setup}

\subsection{Initial Configuration}
Before using Git, configure user identity. This information is associated with every commit.

\subsubsection{Setting Username}
\begin{cmdbox}
\begin{verbatim}
git config --global user.name "Your Name"
\end{verbatim}
\end{cmdbox}
\begin{itemize}[leftmargin=*]
    \item Username appears in all commits
    \item \texttt{--global} applies to all repositories
\end{itemize}

\subsubsection{Setting Email}
\begin{cmdbox}
\begin{verbatim}
git config --global user.email "your.email@example.com"
\end{verbatim}
\end{cmdbox}
\begin{itemize}[leftmargin=*]
    \item Email associates commits with user
    \item Should match email on Git platforms
\end{itemize}

\subsubsection{Verifying Configuration}
\begin{cmdbox}
\begin{verbatim}
git config --global user.name
git config --global user.email
\end{verbatim}
\end{cmdbox}

These commands display currently saved Git configuration values.

\newpage

% ========================
% SECTION 5: BASIC COMMANDS
% ========================
\section{Essential Git Commands}

\subsection{Repository Initialization}

\subsubsection{\texttt{git init}}
Initializes a new Git repository.
\begin{cmdbox}
\begin{verbatim}
git init
\end{verbatim}
\end{cmdbox}
\begin{itemize}[leftmargin=*]
    \item Creates hidden \texttt{.git} directory
    \item Stores version control data
    \item Execute once at project start
\end{itemize}

\subsection{Status and Tracking}

\subsubsection{\texttt{git status}}
Displays current repository state.
\begin{cmdbox}
\begin{verbatim}
git status
\end{verbatim}
\end{cmdbox}
\begin{itemize}[leftmargin=*]
    \item Shows modified, staged, and untracked files
    \item Indicates which changes are ready to commit
    \item Helps understand current working state
\end{itemize}

\subsection{Staging Files}

\subsubsection{\texttt{git add .}}
Stages all modified and untracked files.
\begin{cmdbox}
\begin{verbatim}
git add .
\end{verbatim}
\end{cmdbox}

\subsubsection{\texttt{git add <file\_name>}}
Stages specific file.
\begin{cmdbox}
\begin{verbatim}
git add index.html
\end{verbatim}
\end{cmdbox}

\subsection{Committing Changes}

\subsubsection{\texttt{git commit -m "<message>"}}
Saves staged changes to local repository.
\begin{cmdbox}
\begin{verbatim}
git commit -m "Add homepage design"
\end{verbatim}
\end{cmdbox}
\begin{itemize}[leftmargin=*]
    \item \texttt{-m} flag provides commit message
    \item Creates project snapshot
    \item Message should be descriptive
\end{itemize}

\subsection{Unstaging Files}

\subsubsection{\texttt{git restore --staged <file\_name>}}
Removes file from staging area.
\begin{cmdbox}
\begin{verbatim}
git restore --staged index.html
\end{verbatim}
\end{cmdbox}
\begin{itemize}[leftmargin=*]
    \item Does not delete file or changes
    \item Useful when file staged accidentally
    \item Changes remain in workspace
\end{itemize}

\subsection{The .gitignore File}
Specifies files and folders Git should ignore.
\begin{cmdbox}
\begin{verbatim}
# Example .gitignore entries
node_modules/
*.log
.env
build/
*.tmp
\end{verbatim}
\end{cmdbox}
\begin{itemize}[leftmargin=*]
    \item Prevents tracking of build files, logs, secrets
    \item One pattern per line
    \item Use wildcards for pattern matching
\end{itemize}

\newpage

% ========================
% SECTION 6: VIEWING HISTORY
% ========================
\section{Viewing and Analyzing History}

\subsection{Basic Log Commands}

\subsubsection{\texttt{git log}}
Displays complete commit history.
\begin{cmdbox}
\begin{verbatim}
git log
\end{verbatim}
\end{cmdbox}
\begin{itemize}[leftmargin=*]
    \item Shows full commit details
    \item Includes SHA ID, author, date, message
    \item Press \texttt{q} to exit
\end{itemize}

\subsubsection{\texttt{git log --oneline}}
Displays condensed commit history.
\begin{cmdbox}
\begin{verbatim}
git log --oneline
\end{verbatim}
\end{cmdbox}
\begin{itemize}[leftmargin=*]
    \item One commit per line
    \item Shows shortened SHA and message
    \item Quick history overview
\end{itemize}

\subsubsection{\texttt{git log --stat}}
Shows file change statistics.
\begin{cmdbox}
\begin{verbatim}
git log --stat
\end{verbatim}
\end{cmdbox}
\begin{itemize}[leftmargin=*]
    \item Displays files changed per commit
    \item Shows lines added/deleted
    \item Useful for understanding commit scope
\end{itemize}

\subsubsection{\texttt{git log -p}}
Displays detailed patch information.
\begin{cmdbox}
\begin{verbatim}
git log -p
\end{verbatim}
\end{cmdbox}
\begin{itemize}[leftmargin=*]
    \item Shows line-by-line changes
    \item Full diff for each commit
    \item Useful for detailed code review
\end{itemize}

\subsection{Branch History}

\subsubsection{\texttt{git log --oneline --all}}
Shows history of all branches.
\begin{cmdbox}
\begin{verbatim}
git log --oneline --all
\end{verbatim}
\end{cmdbox}

\subsubsection{\texttt{git log --oneline --all --graph}}
Displays graphical branch representation.
\begin{cmdbox}
\begin{verbatim}
git log --oneline --all --graph
\end{verbatim}
\end{cmdbox}
\begin{itemize}[leftmargin=*]
    \item Visualizes branching structure
    \item Shows merge points
    \item Helps understand development flow
\end{itemize}

\subsection{Viewing Specific Commits}

\subsubsection{\texttt{git show <sha\_id>}}
Displays detailed commit information.
\begin{cmdbox}
\begin{verbatim}
git show a3f8e2
\end{verbatim}
\end{cmdbox}
\begin{itemize}[leftmargin=*]
    \item First 6 digits of SHA sufficient
    \item Shows commit metadata and changes
    \item Includes author, date, diff
\end{itemize}

\subsection{Comparing Changes}

\subsubsection{\texttt{git diff}}
Shows differences between files.
\begin{cmdbox}
\begin{verbatim}
git diff
\end{verbatim}
\end{cmdbox}
\begin{itemize}[leftmargin=*]
    \item Compares workspace vs staging area
    \item Shows unstaged changes
    \item Review before committing
\end{itemize}

\newpage

% ========================
% SECTION 7: BRANCHING
% ========================
\section{Branching and Navigation}

\subsection{Understanding Branches}
Branches are snapshots of a repository allowing independent development:
\begin{itemize}[leftmargin=*]
    \item Main branch contains stable version
    \item Feature branches for new development
    \item Enables parallel work by multiple developers
    \item Changes tested before merging to main
\end{itemize}

\subsection{Branch Commands}

\subsubsection{\texttt{git branch}}
Lists all available branches.
\begin{cmdbox}
\begin{verbatim}
git branch
\end{verbatim}
\end{cmdbox}
\begin{itemize}[leftmargin=*]
    \item Current branch marked with asterisk (*)
    \item Shows local branches only
\end{itemize}

\subsubsection{\texttt{git branch <name>}}
Creates new branch.
\begin{cmdbox}
\begin{verbatim}
git branch feature-login
\end{verbatim}
\end{cmdbox}

\subsubsection{\texttt{git branch <name> <sha\_id>}}
Creates branch at specific commit.
\begin{cmdbox}
\begin{verbatim}
git branch hotfix a3f8e2
\end{verbatim}
\end{cmdbox}
\begin{itemize}[leftmargin=*]
    \item Useful for starting from older commit
    \item Enables targeted development
\end{itemize}

\subsection{Switching Branches}

\subsubsection{\texttt{git checkout <branch>}}
Switches to specified branch.
\begin{cmdbox}
\begin{verbatim}
git checkout feature-login
\end{verbatim}
\end{cmdbox}
\begin{itemize}[leftmargin=*]
    \item Updates workspace to match branch
    \item Changes HEAD pointer
    \item Files may appear/disappear based on branch
\end{itemize}

\subsubsection{\texttt{git checkout <sha\_id>}}
Moves to specific commit.
\begin{cmdbox}
\begin{verbatim}
git checkout a3f8e2
\end{verbatim}
\end{cmdbox}
\begin{itemize}[leftmargin=*]
    \item Enters \textit{detached HEAD} state
    \item Used for inspecting old versions
    \item Not for making new commits
\end{itemize}

\subsubsection{\texttt{git checkout main/master}}
Returns to main branch.
\begin{cmdbox}
\begin{verbatim}
git checkout main
\end{verbatim}
\end{cmdbox}

\subsection{Branch Switching Behavior}
\begin{tcolorbox}[colback=blue!5!white,colframe=blue!75!black,title=Important Note]
When switching branches:
\begin{itemize}
    \item Files from previous branch may be removed from workspace
    \item Files are \textbf{NOT permanently deleted}
    \item They are restored when switching back
    \item Git updates workspace to match selected branch
\end{itemize}
\end{tcolorbox}

\subsection{Deleting Branches}

\subsubsection{\texttt{git branch -d <branch>}}
Deletes merged branch safely.
\begin{cmdbox}
\begin{verbatim}
git branch -d feature-login
\end{verbatim}
\end{cmdbox}

\subsubsection{\texttt{git branch -D <branch>}}
Force deletes branch (even if unmerged).
\begin{cmdbox}
\begin{verbatim}
git branch -D experimental-feature
\end{verbatim}
\end{cmdbox}

\begin{tcolorbox}[colback=red!5!white,colframe=red!75!black,title=Branch Deletion Rule]
A branch currently in use cannot be deleted. Switch to another branch first to avoid data loss.
\end{tcolorbox}

\newpage

% ========================
% SECTION 8: MERGING
% ========================
\section{Merging Branches}

\subsection{The Merge Process}

\subsubsection{\texttt{git merge <branch>}}
Merges specified branch into current branch.
\begin{cmdbox}
\begin{verbatim}
git checkout main
git merge feature-login
\end{verbatim}
\end{cmdbox}
\begin{itemize}[leftmargin=*]
    \item Combines changes from branches
    \item Always merge \textit{into} current branch
    \item May require conflict resolution
\end{itemize}

\subsection{Types of Merging}

\subsubsection{Fast-Forward Merge}
Occurs when current branch has no new commits since branching.
\begin{itemize}[leftmargin=*]
    \item Git simply moves branch pointer forward
    \item No merge commit created
    \item Results in linear history
    \item Clean and simple
\end{itemize}

\textbf{Scenario:}
\begin{verbatim}
main:     A---B
               \
feature:        C---D

After merge:
main:     A---B---C---D
\end{verbatim}

\subsubsection{Complex Merge (Three-Way Merge)}
Occurs when both branches have new commits.
\begin{itemize}[leftmargin=*]
    \item Git creates new merge commit
    \item Combines histories of both branches
    \item May result in merge conflicts
    \item Requires manual conflict resolution
\end{itemize}

\textbf{Scenario:}
\begin{verbatim}
main:     A---B---E---F
               \
feature:        C---D

After merge:
main:     A---B---E---F---M
               \       /
feature:        C---D
\end{verbatim}

\subsection{Branching Strategies}
Common strategies for organizing development:
\begin{itemize}[leftmargin=*]
    \item \textbf{Feature Branching}: One branch per feature
    \item \textbf{Git Flow}: Structured workflow with multiple branch types
    \item \textbf{Release Branching}: Separate branches for releases
    \item \textbf{Trunk-Based}: Short-lived branches, frequent merging
\end{itemize}

\newpage

% ========================
% SECTION 9: REMOTE REPOS
% ========================
\section{Working with Remote Repositories}

\subsection{Connecting to Remote}

\subsubsection{\texttt{git remote add origin <url>}}
Adds remote repository reference.
\begin{cmdbox}
\begin{verbatim}
git remote add origin https://github.com/username/repo.git
\end{verbatim}
\end{cmdbox}
\begin{itemize}[leftmargin=*]
    \item Links local repository to remote server
    \item \texttt{origin} is conventional name
    \item Can have multiple remotes
\end{itemize}

\subsubsection{\texttt{git remote -v}}
Displays all remote repositories.
\begin{cmdbox}
\begin{verbatim}
git remote -v
\end{verbatim}
\end{cmdbox}
\begin{itemize}[leftmargin=*]
    \item Shows remote names and URLs
    \item Displays fetch and push URLs
\end{itemize}

\subsubsection{\texttt{git remote rename <old> <new>}}
Renames existing remote.
\begin{cmdbox}
\begin{verbatim}
git remote rename origin upstream
\end{verbatim}
\end{cmdbox}

\subsection{Pushing Changes}

\subsubsection{\texttt{git push origin <branch>}}
Uploads local commits to remote.
\begin{cmdbox}
\begin{verbatim}
git push origin main
\end{verbatim}
\end{cmdbox}
\begin{itemize}[leftmargin=*]
    \item Synchronizes local with remote
    \item Requires write permissions
    \item Updates remote branch
\end{itemize}

\subsubsection{\texttt{git push origin <branch> --force}}
Forcefully pushes commits, overwriting remote.
\begin{cmdbox}
\begin{verbatim}
git push origin main --force
\end{verbatim}
\end{cmdbox}
\begin{tcolorbox}[colback=red!5!white,colframe=red!75!black,title=Warning]
Use with extreme caution! Force push overwrites remote history and can cause data loss for collaborators.
\end{tcolorbox}

\subsection{Fetching and Pulling}

\subsubsection{\texttt{git clone <url>}}
Creates local copy of remote repository.
\begin{cmdbox}
\begin{verbatim}
git clone https://github.com/username/repo.git
\end{verbatim}
\end{cmdbox}
\begin{itemize}[leftmargin=*]
    \item Downloads complete repository
    \item Automatically sets up remote reference
    \item Creates directory with repo name
\end{itemize}

\subsubsection{\texttt{git fetch}}
Downloads changes without merging.
\begin{cmdbox}
\begin{verbatim}
git fetch origin
\end{verbatim}
\end{cmdbox}
\begin{itemize}[leftmargin=*]
    \item Updates remote-tracking branches
    \item Does not modify working directory
    \item Safe operation for reviewing changes
\end{itemize}

\subsubsection{\texttt{git pull origin <branch>}}
Fetches and automatically merges changes.
\begin{cmdbox}
\begin{verbatim}
git pull origin main
\end{verbatim}
\end{cmdbox}
\begin{itemize}[leftmargin=*]
    \item Equivalent to \texttt{git fetch} + \texttt{git merge}
    \item Updates current branch
    \item May cause merge conflicts
\end{itemize}

\subsection{Understanding Origin and Upstream}
\begin{itemize}[leftmargin=*]
    \item \textbf{origin}: User's fork/personal repository
    \item \textbf{upstream}: Original repository (in fork workflow)
    \item \textbf{main}: Default primary branch name
    \item \textbf{HEAD}: Pointer to current branch/commit
\end{itemize}

\newpage

% ========================
% SECTION 10: ADVANCED OPS
% ========================
\section{Advanced Operations}

\subsection{Reverting Changes}

\subsubsection{\texttt{git revert <sha\_id>}}
Creates new commit that reverses changes.
\begin{cmdbox}
\begin{verbatim}
git revert a3f8e2
\end{verbatim}
\end{cmdbox}
\begin{itemize}[leftmargin=*]
    \item Does not delete commit history
    \item Safer than reset in shared repos
    \item Preserves project timeline
    \item Recommended for public branches
\end{itemize}

\subsection{Viewing Differences with Remote}

\subsubsection{\texttt{git log upstream/main --oneline}}
Displays upstream commit history.
\begin{cmdbox}
\begin{verbatim}
git log upstream/main --oneline
\end{verbatim}
\end{cmdbox}

\subsubsection{\texttt{git diff upstream/main}}
Shows differences with upstream.
\begin{cmdbox}
\begin{verbatim}
git diff upstream/main
\end{verbatim}
\end{cmdbox}
\begin{itemize}[leftmargin=*]
    \item Compares current branch with upstream
    \item Helps identify sync status
    \item Useful before creating pull requests
\end{itemize}

\subsection{Syncing with Upstream}

\subsubsection{\texttt{git pull upstream main}}
Fetches and merges from original repository.
\begin{cmdbox}
\begin{verbatim}
git pull upstream main
\end{verbatim}
\end{cmdbox}
\begin{itemize}[leftmargin=*]
    \item Updates fork with original changes
    \item Essential for staying current
    \item Prevents merge conflicts
\end{itemize}

\subsubsection{\texttt{git remote add upstream <url>}}
Adds original repository as upstream.
\begin{cmdbox}
\begin{verbatim}
git remote add upstream https://github.com/original/repo.git
\end{verbatim}
\end{cmdbox}

\newpage

% ========================
% SECTION 11: PLATFORMS
% ========================
\section{Git Hosting Platforms}

\subsection{GitHub}
\textbf{GitHub} is a web-based hosting service for Git repositories.
\begin{itemize}[leftmargin=*]
    \item Owned by Microsoft
    \item Largest developer community
    \item Provides collaboration tools:
    \begin{itemize}
        \item Pull requests
        \item Issue tracking
        \item Project boards
        \item GitHub Actions (CI/CD)
        \item GitHub Pages (hosting)
    \end{itemize}
    \item Free public and private repositories
    \item Popular for open-source projects
\end{itemize}

\subsection{GitLab}
\textbf{GitLab} is a complete DevOps platform.
\begin{itemize}[leftmargin=*]
    \item Self-hosted or cloud-based
    \item Integrated CI/CD pipelines
    \item Built-in container registry
    \item Issue tracking and boards
    \item Wiki and documentation
    \item Security scanning features
    \item Can be deployed on-premises
\end{itemize}

\subsection{SSH Protocol}
\textbf{SSH} (Secure Shell) provides secure authentication.
\begin{itemize}[leftmargin=*]
    \item Encrypted communication protocol
    \item Password-less authentication via keys
    \item More secure than HTTPS with passwords
    \item Required for many Git operations
    \item SSH keys consist of public/private pair
\end{itemize}

\newpage

% ========================
% SECTION 12: WORKFLOWS
% ========================
\section{Fork and Pull Request Workflow}

\subsection{Understanding Forks}
A \textbf{fork} is a personal copy of another user's repository.
\begin{itemize}[leftmargin=*]
    \item Allows independent development
    \item Does not affect original repository
    \item Common in open-source projects
    \item Changes proposed via pull requests
\end{itemize}

\subsection{Clone vs Fork}
\begin{center}
\begin{tabular}{|p{0.45\textwidth}|p{0.45\textwidth}|}
\hline
\textbf{Cloning} & \textbf{Forking} \\
\hline
Creates local copy of repository & Creates copy in your GitHub account \\
\hline
Directly linked to original & Independent copy on remote \\
\hline
Used for any repository access & Used for contributing to others' projects \\
\hline
Command: \texttt{git clone <url>} & Done via GitHub/GitLab interface \\
\hline
\end{tabular}
\end{center}

\subsection{Pull Requests}
A \textbf{pull request} (PR) proposes changes for review.
\begin{itemize}[leftmargin=*]
    \item Requests code review and approval
    \item Allows discussion about changes
    \item Ensures code quality before merging
    \item Can be approved, rejected, or requested changes
    \item Commonly used workflow in teams
\end{itemize}

\subsection{Git Fork Workflow}
\begin{enumerate}[leftmargin=*]
    \item \textbf{Fork} the original repository on GitHub
    \item \textbf{Clone} your fork locally:
    \begin{cmdbox}
    \texttt{git clone https://github.com/yourname/repo.git}
    \end{cmdbox}
    \item \textbf{Add upstream} remote:
    \begin{cmdbox}
    \texttt{git remote add upstream https://github.com/original/repo.git}
    \end{cmdbox}
    \item \textbf{Create feature branch}:
    \begin{cmdbox}
    \texttt{git checkout -b feature-name}
    \end{cmdbox}
    \item \textbf{Make changes} and commit
    \item \textbf{Push to origin}:
    \begin{cmdbox}
    \texttt{git push origin feature-name}
    \end{cmdbox}
    \item \textbf{Create pull request} from your fork to upstream/main
    \item \textbf{Wait for review} and address feedback
    \item After approval, changes are \textbf{merged} by maintainer
\end{enumerate}

\subsection{Keeping Fork Updated}
\begin{cmdbox}
\begin{verbatim}
# Fetch upstream changes
git fetch upstream

# Merge upstream/main into your local main
git checkout main
git merge upstream/main

# Push updates to your fork
git push origin main
\end{verbatim}
\end{cmdbox}

\newpage

% ========================
% SECTION 13: TERMINAL
% ========================
\section{Terminal and Directory Commands}

\subsection{Basic Directory Operations}

\subsubsection{\texttt{mkdir}}
Creates new directory.
\begin{cmdbox}
\begin{verbatim}
mkdir project-folder
\end{verbatim}
\end{cmdbox}

\subsubsection{\texttt{cd}}
Changes current directory.
\begin{cmdbox}
\begin{verbatim}
cd project-folder      # Enter directory
cd ..                  # Go up one level
cd ~                   # Go to home directory
\end{verbatim}
\end{cmdbox}

\subsection{Terminal Control Commands}

\subsubsection{\texttt{cls} (Command Prompt)}
Clears Command Prompt screen.
\begin{cmdbox}
\begin{verbatim}
cls
\end{verbatim}
\end{cmdbox}

\subsubsection{\texttt{clear} (PowerShell/Bash)}
Clears terminal screen.
\begin{cmdbox}
\begin{verbatim}
clear
\end{verbatim}
\end{cmdbox}

\subsubsection{\texttt{q}}
Quits paginated output (git log, help pages).
\begin{cmdbox}
\begin{verbatim}
q
\end{verbatim}
\end{cmdbox}

\newpage

% ========================
% SECTION 14: GLOSSARY
% ========================
\section{Comprehensive Glossary}

\begin{description}[leftmargin=3cm,style=nextline]
    \item[Branch] A separate line of development allowing developers to work on features independently without affecting the main codebase.
    
    \item[Clone] A local copy of a remote Git repository created on a developer's computer, including all history and branches.
    
    \item[Commit] A snapshot of the project's state at a specific point, with a message describing changes made.
    
    \item[Continuous Delivery (CD)] Automated software movement through development lifecycle, ensuring code can be released reliably anytime.
    
    \item[Continuous Integration (CI)] Practice where developers frequently integrate code changes into shared codebase, often multiple times daily.
    
    \item[Distributed Version Control System (DVCS)] System tracking code changes regardless of storage location. Each user has complete repository copy.
    
    \item[Fork] Copy of existing repository in user's GitHub account, allowing independent development without affecting original.
    
    \item[Git] Free, open-source distributed version control system under GNU General Public License. Enables local project maintenance worldwide.
    
    \item[GitHub] Web-hosted platform providing Git repository hosting and collaboration tools like pull requests and issue tracking.
    
    \item[GitHub Branches] Store all repository files, used to isolate code changes. Completed changes merge back to main branch.
    
    \item[GitLab] Complete DevOps platform offering Git repository hosting, source code management, and CI/CD tools in single application.
    
    \item[HEAD] Pointer to currently checked-out branch or commit. Represents current working position in repository.
    
    \item[Merge] Process combining changes from one branch into another, typically merging feature branch into main.
    
    \item[Origin] Default name for remote repository. Points to main remote location where code is pushed/pulled.
    
    \item[Pull Request] Process requesting review and approval of code changes before merging into main branch.
    
    \item[Repository] Data structure storing project files, folders, and version history. Configured for version control.
    
    \item[SSH Protocol] Secure method for remote login and communication, commonly used for authenticating Git operations.
    
    \item[Staging Area] Intermediate preparation zone where changes are prepared before committing to repository.
    
    \item[Upstream] Remote reference to original repository (used in fork workflows). Distinguished from personal fork (origin).
    
    \item[Version Control] System tracking file changes over time, allowing recovery of previous versions if errors occur.
    
    \item[Working Directory] Directory on local file system containing files and subdirectories associated with Git repository.
\end{description}

\newpage

% ========================
% SECTION 15: QUICK REF
% ========================
\section{Quick Reference Guide}

\subsection{Essential Commands Summary}

\begin{tcolorbox}[colback=green!5!white,colframe=green!75!black,title=Setup \& Configuration]
\begin{verbatim}
git config --global user.name "Name"
git config --global user.email "email@example.com"
git init
\end{verbatim}
\end{tcolorbox}

\begin{tcolorbox}[colback=blue!5!white,colframe=blue!75!black,title=Basic Workflow]
\begin{verbatim}
git status
git add <file>              # Stage specific file
git add .                   # Stage all changes
git commit -m "message"
git push origin main
\end{verbatim}
\end{tcolorbox}

\begin{tcolorbox}[colback=purple!5!white,colframe=purple!75!black,title=Branching]
\begin{verbatim}
git branch                  # List branches
git branch <name>           # Create branch
git checkout <branch>       # Switch branch
git merge <branch>          # Merge branch
git branch -d <branch>      # Delete merged branch
git branch -D <branch>      # Force delete branch
\end{verbatim}
\end{tcolorbox}

\begin{tcolorbox}[colback=orange!5!white,colframe=orange!75!black,title=Remote Operations]
\begin{verbatim}
git clone <url>
git remote add origin <url>
git remote -v
git pull origin main
git push origin main
git fetch origin
\end{verbatim}
\end{tcolorbox}

\begin{tcolorbox}[colback=red!5!white,colframe=red!75!black,title=History \& Inspection]
\begin{verbatim}
git log
git log --oneline
git log --oneline --all --graph
git show <sha-id>
git diff
\end{verbatim}
\end{tcolorbox}

\begin{tcolorbox}[colback=yellow!5!white,colframe=yellow!75!black,title=Undoing Changes]
\begin{verbatim}
git restore --staged <file>  # Unstage file
git revert <sha-id>          # Revert commit
git checkout <sha-id>        # View old commit
\end{verbatim}
\end{tcolorbox}

\newpage

\subsection{Common Workflows}

\subsubsection{Starting New Project}
\begin{enumerate}
    \item Create directory: \texttt{mkdir project-name}
    \item Navigate: \texttt{cd project-name}
    \item Initialize: \texttt{git init}
    \item Create files and make changes
    \item Stage: \texttt{git add .}
    \item Commit: \texttt{git commit -m "Initial commit"}
    \item Add remote: \texttt{git remote add origin <url>}
    \item Push: \texttt{git push origin main}
\end{enumerate}

\subsubsection{Contributing to Existing Project}
\begin{enumerate}
    \item Fork repository on GitHub
    \item Clone your fork: \texttt{git clone <your-fork-url>}
    \item Add upstream: \texttt{git remote add upstream <original-url>}
    \item Create branch: \texttt{git checkout -b feature-name}
    \item Make changes and commit
    \item Push to fork: \texttt{git push origin feature-name}
    \item Create pull request on GitHub
\end{enumerate}

\subsubsection{Keeping Fork Updated}
\begin{enumerate}
    \item Fetch upstream: \texttt{git fetch upstream}
    \item Switch to main: \texttt{git checkout main}
    \item Merge upstream: \texttt{git merge upstream/main}
    \item Push to fork: \texttt{git push origin main}
\end{enumerate}

\subsection{Best Practices}
\begin{itemize}[leftmargin=*]
    \item \textbf{Commit often} with meaningful messages
    \item \textbf{Pull before push} to avoid conflicts
    \item \textbf{Use branches} for features and fixes
    \item \textbf{Review changes} before committing
    \item \textbf{Write clear commit messages} (50 char summary, detailed description)
    \item \textbf{Keep commits atomic} (one logical change per commit)
    \item \textbf{Use .gitignore} to exclude unnecessary files
    \item \textbf{Test before committing} to maintain working code
    \item \textbf{Avoid force push} on shared branches
    \item \textbf{Document your code} and repository
\end{itemize}

\subsection{Commit Message Guidelines}
\begin{tcolorbox}[colback=gray!5!white,colframe=gray!75!black,title=Good Commit Messages]
\textbf{Format:}
\begin{verbatim}
Short summary (50 chars or less)

More detailed explanation if needed. Wrap at 72 characters.
Explain what and why, not how.

- Bullet points okay
- Use present tense: "Add feature" not "Added feature"
\end{verbatim}

\textbf{Examples:}
\begin{itemize}
    \item \texttt{Add user authentication system}
    \item \texttt{Fix bug in payment processing}
    \item \texttt{Refactor database connection logic}
    \item \texttt{Update README with installation steps}
\end{itemize}
\end{tcolorbox}

\newpage

% ========================
% CONCLUSION
% ========================
\section{Conclusion}

Git is an essential tool in modern software development that enables:
\begin{itemize}[leftmargin=*]
    \item \textbf{Effective version control} with complete history tracking
    \item \textbf{Smooth collaboration} among distributed teams
    \item \textbf{Non-linear development} through branching and merging
    \item \textbf{Code quality} through review processes
    \item \textbf{Project management} via platforms like GitHub and GitLab
\end{itemize}

\vspace{1em}

Mastering Git requires practice and understanding of its core concepts. This reference guide provides the foundation needed to work effectively with Git in professional software development environments.

\vspace{1em}

\begin{tcolorbox}[colback=blue!5!white,colframe=blue!75!black,title=Remember]
\begin{itemize}
    \item Git is distributed - every copy is a full backup
    \item Commits are permanent - think before you commit
    \item Branches are cheap - use them liberally
    \item Communication is key - write clear commit messages
    \item When in doubt, consult \texttt{git --help}
\end{itemize}
\end{tcolorbox}

\vspace{2em}

\subsection{Additional Resources}
\begin{itemize}[leftmargin=*]
    \item Official Git Documentation: \url{https://git-scm.com/doc}
    \item GitHub Guides: \url{https://guides.github.com}
    \item GitLab Documentation: \url{https://docs.gitlab.com}
    \item Interactive Git Tutorial: \url{https://learngitbranching.js.org}
    \item Git Cheat Sheet: \url{https://education.github.com/git-cheat-sheet-education.pdf}
\end{itemize}

\vspace{2em}
\hrule
\vspace{0.5em}
\begin{center}
\textit{End of Git \& GitHub Complete Reference Guide}
\end{center}

\end{document}