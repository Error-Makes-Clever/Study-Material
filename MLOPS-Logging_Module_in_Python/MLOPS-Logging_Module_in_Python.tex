\documentclass[11pt,a4paper]{article}
\usepackage[margin=1in]{geometry}
\usepackage{amsmath}
\usepackage{amssymb}
\usepackage{titlesec}
\usepackage{enumitem}
\usepackage{xcolor}
\usepackage[most]{tcolorbox}
\usepackage{fancyhdr}
\usepackage{listings}
\usepackage{hyperref}

\usepackage{url}  % Add this for better URL breaking

% Configure hyperref
\hypersetup{
    breaklinks=true,
    colorlinks=true,
    linkcolor=blue,
    urlcolor=blue
}

\usepackage{graphicx}

% Fix header height
\setlength{\headheight}{14pt}

% Header and Footer
\pagestyle{fancy}
\fancyhf{}
\rhead{Python Logging Complete Guide}
\lhead{MLOps Reference}
\cfoot{\thepage}

% Title formatting
\titleformat{\section}{\Large\bfseries\color{blue!70!black}}{\thesection}{1em}{}[\titlerule]
\titleformat{\subsection}{\large\bfseries\color{blue!50!black}}{\thesubsection}{1em}{}
\titleformat{\subsubsection}{\normalsize\bfseries\color{blue!40!black}}{\thesubsubsection}{1em}{}

% Code styling
\definecolor{codebg}{gray}{0.95}
\definecolor{codegreen}{rgb}{0,0.6,0}
\definecolor{codegray}{rgb}{0.5,0.5,0.5}
\definecolor{codepurple}{rgb}{0.58,0,0.82}

\lstdefinestyle{pythonstyle}{
    language=Python,
    backgroundcolor=\color{codebg},
    commentstyle=\color{codegreen},
    keywordstyle=\color{blue},
    numberstyle=\tiny\color{codegray},
    stringstyle=\color{codepurple},
    basicstyle=\ttfamily\small,
    breaklines=true,
    captionpos=b,
    keepspaces=true,
    numbers=left,
    numbersep=5pt,
    showspaces=false,
    showstringspaces=false,
    showtabs=false,
    tabsize=4,
    frame=single,
    xleftmargin=2em,
    framexleftmargin=1.5em
}

\lstset{style=pythonstyle}

% Command box
\newtcolorbox{cmdbox}{
    colback=codebg,
    colframe=black!50,
    boxrule=0.5pt,
    left=2mm,
    right=2mm,
    top=1mm,
    bottom=1mm,
    breakable
}

% Example box
\newtcolorbox{examplebox}[1]{
    colback=green!5!white,
    colframe=green!75!black,
    title=#1,
    fonttitle=\bfseries,
    breakable,
    enhanced jigsaw
}

% Note box
\newtcolorbox{notebox}{
    colback=yellow!10!white,
    colframe=orange!75!black,
    title=Important Note,
    fonttitle=\bfseries,
    breakable,
    enhanced jigsaw
}

% Warning box
\newtcolorbox{warningbox}{
    colback=red!5!white,
    colframe=red!75!black,
    title=Warning,
    fonttitle=\bfseries,
    breakable,
    enhanced jigsaw
}

% Info box - FIXED: Added missing comma
\newtcolorbox{infobox}[1]{
    colback=blue!5!white,
    colframe=blue!75!black,
    title=#1,
    fonttitle=\bfseries,
    breakable,
    enhanced jigsaw
}

\begin{document}

% Title Page
\begin{titlepage}
    \centering
    \vspace*{2cm}
    {\Huge\bfseries Python Logging Module\\[0.5cm] Complete Reference Guide\par}
    \vspace{1cm}
    {\Large For MLOps and Production Systems\par}
    \vspace{2cm}
    {\large A Comprehensive Guide to Logging,\\
    Best Practices, and Real-World Implementation\par}
    \vspace{3cm}
    {\Large\bfseries Sujil S\par}
    \vspace{0.5cm}
    {\large\texttt{sujil9480@gmail.com}\par}
    \vfill
    {\large \today\par}
\end{titlepage}

\tableofcontents
\newpage

% ========================
% SECTION 1: INTRODUCTION
% ========================
\section{Introduction to Python Logging}

\subsection{What is Logging?}

\textbf{Logging} is the process of recording events, messages, and information during program execution. It provides a way to track what happens when software runs, which is essential for understanding application behavior, debugging issues, and monitoring systems in production.

\subsubsection{Why Logging Matters}

Logging is critical for:
\begin{itemize}[leftmargin=*]
    \item \textbf{Debugging and Troubleshooting}: Identify and fix bugs efficiently
    \item \textbf{Monitoring}: Track application behavior in production
    \item \textbf{Auditing}: Maintain records for compliance and security
    \item \textbf{Performance Analysis}: Identify bottlenecks and optimization opportunities
    \item \textbf{Understanding Flow}: Trace execution path through complex systems
\end{itemize}

\subsection{Why Use Logging in MLOps?}

In Machine Learning Operations (MLOps), logging becomes even more critical:

\begin{enumerate}[leftmargin=*]
    \item \textbf{Model Training Monitoring}:
    \begin{itemize}
        \item Track training epochs and iterations
        \item Record loss values and metrics
        \item Monitor convergence and performance
        \item Log hyperparameters for reproducibility
    \end{itemize}
    
    \item \textbf{Data Pipeline Tracking}:
    \begin{itemize}
        \item Monitor data ingestion processes
        \item Log preprocessing steps
        \item Track data quality issues
        \item Record transformation operations
    \end{itemize}
    
    \item \textbf{Model Serving}:
    \begin{itemize}
        \item Log prediction requests and responses
        \item Track API performance
        \item Monitor model confidence scores
        \item Record error rates
    \end{itemize}
    
    \item \textbf{Error Tracking}:
    \begin{itemize}
        \item Identify failures quickly
        \item Debug production issues
        \item Track error patterns
        \item Generate alerts for critical errors
    \end{itemize}
    
    \item \textbf{Performance Monitoring}:
    \begin{itemize}
        \item Track inference times
        \item Monitor resource usage (CPU, GPU, memory)
        \item Identify performance degradation
        \item Optimize system bottlenecks
    \end{itemize}
    
    \item \textbf{Reproducibility}:
    \begin{itemize}
        \item Maintain audit trails for experiments
        \item Document model versions and configurations
        \item Track data versions used
        \item Enable experiment reproduction
    \end{itemize}
\end{enumerate}

\subsection{Logging vs Print Statements}

Many beginners use \texttt{print()} statements for debugging. However, in production code, the logging module is vastly superior:

\begin{center}
\begin{tabular}{|p{0.45\textwidth}|p{0.45\textwidth}|}
\hline
\textbf{print() Statements} & \textbf{logging Module} \\
\hline
Always outputs to console & Flexible output destinations (file, console, remote) \\
\hline
No severity levels & Five severity levels (DEBUG to CRITICAL) \\
\hline
Difficult to filter messages & Easy filtering by level \\
\hline
No automatic timestamps & Automatic timestamps available \\
\hline
Cannot be disabled easily & Can be enabled/disabled by configuration \\
\hline
Not suitable for production & Production-ready and battle-tested \\
\hline
No structured format & Consistent, structured format \\
\hline
Limited context information & Rich context (filename, line number, function) \\
\hline
Performance impact on I/O & Optimized performance with lazy evaluation \\
\hline
No rotation capabilities & Automatic log file rotation \\
\hline
\end{tabular}
\end{center}

\begin{warningbox}
\textbf{Production Code Rule}: Never use \texttt{print()} statements in production code. Always use the logging module for better control, flexibility, and maintainability.
\end{warningbox}

\newpage

% ========================
% SECTION 2: CORE CONCEPTS
% ========================
\section{Core Logging Components}

\subsection{The Logging Architecture}

The Python logging module follows a hierarchical architecture with four main components that work together to provide flexible and powerful logging capabilities.

\begin{enumerate}[leftmargin=*]
    \item \textbf{Loggers}: Entry point for logging messages in your application
    \item \textbf{Handlers}: Determine where log records are sent (console, file, network)
    \item \textbf{Formatters}: Define the layout and content of log messages
    \item \textbf{Filters}: Provide fine-grained control over which log records are processed
\end{enumerate}

\subsection{Logger Objects}

The \textbf{Logger} is the primary interface that applications use to write log messages.

\subsubsection{Creating a Logger}

\begin{lstlisting}
import logging

# Create a logger with module name (recommended best practice)
logger = logging.getLogger(__name__)

# Create a logger with custom name
logger = logging.getLogger('my_application')

# Get the root logger
root_logger = logging.getLogger()
\end{lstlisting}

\begin{notebox}
\textbf{Best Practice}: Always use \texttt{\_\_name\_\_} when creating loggers. This creates a hierarchical naming structure based on your module organization, making it easier to manage logging across large applications.
\end{notebox}

\subsubsection{Logger Methods}

Loggers provide methods for different severity levels:

\begin{itemize}[leftmargin=*]
    \item \texttt{logger.debug(msg)} - Detailed diagnostic information for debugging
    \item \texttt{logger.info(msg)} - General informational messages about normal operation
    \item \texttt{logger.warning(msg)} - Warning messages indicating potential issues
    \item \texttt{logger.error(msg)} - Error messages for serious problems
    \item \texttt{logger.critical(msg)} - Critical error messages for severe failures
    \item \texttt{logger.exception(msg)} - Error with full traceback (use in except blocks)
\end{itemize}

\subsection{Handler Objects}

\textbf{Handlers} determine where log messages are sent. Multiple handlers can be attached to a single logger.

\subsubsection{Common Handler Types}

\begin{description}[leftmargin=4cm,style=nextline]
    \item[StreamHandler] Sends logs to console (stdout/stderr) - useful for development
    \item[FileHandler] Writes logs to a file - basic file logging
    \item[RotatingFileHandler] Rotates log files based on size - prevents files from growing too large
    \item[TimedRotatingFileHandler] Rotates log files based on time intervals - daily, weekly, etc.
    \item[HTTPHandler] Sends logs to HTTP server - for centralized logging
    \item[SMTPHandler] Sends logs via email - for critical alerts
    \item[SysLogHandler] Sends logs to Unix syslog daemon
    \item[SocketHandler] Sends logs over network socket
    \item[QueueHandler] Sends logs to queue for async processing
\end{description}

\subsection{Formatter Objects}

\textbf{Formatters} specify the layout and content of log messages, allowing you to customize how information is displayed.

\subsubsection{Common Format Attributes}

The following placeholders can be used in format strings:

\begin{itemize}[leftmargin=*]
    \item \texttt{\%(asctime)s} - Human-readable timestamp of log record
    \item \texttt{\%(name)s} - Name of the logger
    \item \texttt{\%(levelname)s} - Logging level (DEBUG, INFO, WARNING, ERROR, CRITICAL)
    \item \texttt{\%(message)s} - The actual log message
    \item \texttt{\%(filename)s} - Source filename where log was called
    \item \texttt{\%(lineno)d} - Line number in source code
    \item \texttt{\%(funcName)s} - Function name where log was called
    \item \texttt{\%(process)d} - Process ID
    \item \texttt{\%(thread)d} - Thread ID
    \item \texttt{\%(pathname)s} - Full pathname of source file
    \item \texttt{\%(module)s} - Module name (filename without extension)
    \item \texttt{\%(levelno)s} - Numeric logging level
\end{itemize}

\subsection{Filter Objects}

\textbf{Filters} provide fine-grained control over which log records are processed. They can be attached to both loggers and handlers.

\vspace{0.5em}

Filters are useful for:
\begin{itemize}[leftmargin=*]
    \item Filtering by specific attributes
    \item Adding contextual information
    \item Implementing complex filtering logic
    \item Rate limiting log messages
\end{itemize}

\newpage

% ========================
% SECTION 3: LOGGING LEVELS
% ========================
\section{Logging Levels}

\subsection{The Five Standard Levels}

Python's logging module defines five severity levels in ascending order of importance. Each level has a numeric value:

\begin{center}
\begin{tabular}{|c|l|c|}
\hline
\textbf{Level Name} & \textbf{Numeric Value} & \textbf{Purpose} \\
\hline
DEBUG & 10 & Detailed diagnostic information \\
\hline
INFO & 20 & Informational messages \\
\hline
WARNING & 30 & Warning messages \\
\hline
ERROR & 40 & Error messages \\
\hline
CRITICAL & 50 & Critical error messages \\
\hline
\end{tabular}
\end{center}

\subsection{DEBUG Level (10)}

\textbf{Purpose}: Most detailed information for diagnosing problems during development.

\vspace{0.5em}

\textbf{When to Use}:
\begin{itemize}[leftmargin=*]
    \item Development and debugging phases
    \item Tracking variable values and state
    \item Understanding detailed program flow
    \item Loop iterations and function entry/exit
    \item Detailed API request/response information
\end{itemize}

\begin{examplebox}{DEBUG Level Examples in MLOps}
\begin{lstlisting}
logger.debug(f"Loading dataset from {data_path}")
logger.debug(f"Feature shape: {X.shape}, Target shape: {y.shape}")
logger.debug(f"Model parameters: {model.get_params()}")
logger.debug(f"Batch {batch_num}: input_shape={input_data.shape}")
logger.debug(f"GPU memory allocated: {torch.cuda.memory_allocated()}")
\end{lstlisting}
\end{examplebox}

\subsection{INFO Level (20)}

\textbf{Purpose}: Confirmation that things are working as expected.

\vspace{0.5em}

\textbf{When to Use}:
\begin{itemize}[leftmargin=*]
    \item Application startup and shutdown
    \item Successful completion of operations
    \item Milestone achievements
    \item Configuration information
    \item Normal operational messages
\end{itemize}

\begin{examplebox}{INFO Level Examples in MLOps}
\begin{lstlisting}
logger.info("Model training started")
logger.info(f"Epoch {epoch}/{total_epochs}: loss={loss:.4f}, accuracy={acc:.4f}")
logger.info("Model saved successfully to disk")
logger.info(f"API server started on port {port}")
logger.info(f"Loaded {len(dataset)} samples from database")
logger.info("Data preprocessing pipeline completed")
\end{lstlisting}
\end{examplebox}

\subsection{WARNING Level (30)}

\textbf{Purpose}: Indication of unexpected events or potential problems that don't prevent the program from working.

\vspace{0.5em}

\textbf{When to Use}:
\begin{itemize}[leftmargin=*]
    \item Using deprecated features
    \item Resource constraints or limitations
    \item Recoverable errors
    \item Configuration issues that have fallbacks
    \item Performance concerns
\end{itemize}

\begin{examplebox}{WARNING Level Examples in MLOps}
\begin{lstlisting}
logger.warning("Missing values detected in dataset")
logger.warning(f"GPU memory usage at {usage}%, close to limit")
logger.warning("Model confidence below threshold: 0.6")
logger.warning("Using default hyperparameters")
logger.warning("Training data smaller than recommended minimum")
logger.warning("Deprecated model architecture detected")
\end{lstlisting}
\end{examplebox}

\subsection{ERROR Level (40)}

\textbf{Purpose}: Serious problems that prevented a specific function or operation from executing.
\vspace{0.5em}

\textbf{When to Use}:
\begin{itemize}[leftmargin=*]
    \item Failed operations
    \item Exception handling (non-critical)
    \item Resource unavailability
    \item Data validation failures
    \item API request failures
\end{itemize}

\begin{examplebox}{ERROR Level Examples in MLOps}
\begin{lstlisting}
logger.error(f"Failed to load model from {model_path}")
logger.error("Database connection failed")
logger.error(f"Invalid input shape: expected {exp_shape}, got {actual_shape}")
logger.error("Prediction API returned error 500")
logger.error("Data quality check failed: too many outliers")
logger.error(f"Feature extraction failed for {feature_name}")
\end{lstlisting}
\end{examplebox}

\subsection{CRITICAL Level (50)}

\textbf{Purpose}: Very serious errors indicating that the program may not be able to continue running.

\vspace{0.45em}

\textbf{When to Use}:
\begin{itemize}[leftmargin=*]
    \item System crashes or imminent crashes
    \item Data corruption
    \item Security breaches
    \item Unrecoverable errors
    \item Service unavailability
\end{itemize}

\begin{examplebox}{CRITICAL Level Examples in MLOps}
\begin{lstlisting}
logger.critical("Out of memory - cannot continue training")
logger.critical("Model serving endpoint unreachable")
logger.critical("Critical security vulnerability detected")
logger.critical("Data pipeline completely failed")
logger.critical("Unable to connect to required external service")
logger.critical("System configuration corrupted")
\end{lstlisting}
\end{examplebox}

\subsection{Setting Logging Levels}

You can set logging levels at both the logger and handler level:

\begin{lstlisting}
import logging

logger = logging.getLogger(__name__)

# Set logger level (processes messages at this level and above)
logger.setLevel(logging.DEBUG)

# Set handler level (outputs messages at this level and above)
handler = logging.StreamHandler()
handler.setLevel(logging.WARNING)

logger.addHandler(handler)

# These will be processed but not output (handler level is WARNING)
logger.debug("Debug message")   # Not output
logger.info("Info message")     # Not output

# These will be both processed and output
logger.warning("Warning message")  # Output
logger.error("Error message")      # Output
\end{lstlisting}

\begin{infobox}{Level Hierarchy}
\textbf{How Levels Work}:
\begin{itemize}[leftmargin=*]
    \item Logger set to INFO will process INFO, WARNING, ERROR, CRITICAL (but not DEBUG)
    \item Handler set to ERROR will only output ERROR and CRITICAL messages
    \item Messages below the set level are completely ignored
    \item Both logger AND handler must allow a level for it to be output
\end{itemize}
\end{infobox}

\newpage

% ========================
% SECTION 4: HANDLERS
% ========================
\section{Handlers in Detail}

\subsection{StreamHandler (Console Handler)}

The \textbf{StreamHandler} sends log output to streams like \texttt{sys.stdout} or \texttt{sys.stderr}. This is the most common handler for development.

\subsubsection{Basic Usage}

\begin{lstlisting}
import logging

logger = logging.getLogger(__name__)
logger.setLevel(logging.DEBUG)

# Create console handler
console_handler = logging.StreamHandler()
console_handler.setLevel(logging.INFO)

# Create formatter
formatter = logging.Formatter(
    '%(asctime)s - %(name)s - %(levelname)s - %(message)s'
)
console_handler.setFormatter(formatter)

# Add handler to logger
logger.addHandler(console_handler)

# Use the logger
logger.info("This appears in console")
logger.debug("This won't appear (below handler level)")
\end{lstlisting}

\subsubsection{Directing to stderr}

\begin{lstlisting}
import sys
import logging

# Direct to stderr instead of stdout
error_handler = logging.StreamHandler(sys.stderr)
error_handler.setLevel(logging.ERROR)

logger.addHandler(error_handler)
\end{lstlisting}

\subsection{FileHandler}

The \textbf{FileHandler} writes log messages to a specified file on disk.

\subsubsection{Basic Usage}

\begin{lstlisting}
import logging

logger = logging.getLogger(__name__)
logger.setLevel(logging.DEBUG)

# Create file handler
file_handler = logging.FileHandler('application.log')
file_handler.setLevel(logging.ERROR)

formatter = logging.Formatter(
    '%(asctime)s - %(levelname)s - %(message)s'
)
file_handler.setFormatter(formatter)

logger.addHandler(file_handler)

logger.error("This is written to application.log")
logger.info("This is not written (below handler level)")
\end{lstlisting}

\subsubsection{File Modes}

\begin{itemize}[leftmargin=*]
    \item \texttt{'w'} - Write mode: Overwrites existing file
    \item \texttt{'a'} - Append mode: Appends to existing file (default)
\end{itemize}

\begin{lstlisting}
# Overwrite mode - starts fresh each time
file_handler = logging.FileHandler('app.log', mode='w')

# Append mode - continues adding to existing file
file_handler = logging.FileHandler('app.log', mode='a')
\end{lstlisting}

\subsection{RotatingFileHandler}

The \textbf{RotatingFileHandler} automatically rotates log files when they reach a certain size. This is essential for production systems to prevent log files from consuming too much disk space.

\subsubsection{Usage Example}

\begin{lstlisting}
from logging.handlers import RotatingFileHandler
import logging

logger = logging.getLogger(__name__)
logger.setLevel(logging.DEBUG)

# Rotate after 10MB, keep 5 backup files
rotating_handler = RotatingFileHandler(
    'app.log',
    maxBytes=10*1024*1024,  # 10 MB
    backupCount=5
)

formatter = logging.Formatter(
    '%(asctime)s - %(name)s - %(levelname)s - %(message)s'
)
rotating_handler.setFormatter(formatter)
logger.addHandler(rotating_handler)

logger.info("Logging with rotation enabled")
\end{lstlisting}

\textbf{How File Rotation Works}:
\begin{enumerate}[leftmargin=*]
    \item When \texttt{app.log} reaches 10MB, it's renamed to \texttt{app.log.1}
    \item A new \texttt{app.log} file is created for current logging
    \item On next rotation: \texttt{app.log.1} $\rightarrow$ \texttt{app.log.2}, new log $\rightarrow$ \texttt{app.log.1}
    \item This continues until \texttt{backupCount} is reached
    \item Oldest file is deleted when new rotation occurs beyond \texttt{backupCount}
\end{enumerate}

\begin{notebox}
\textbf{Production Tip}: For production systems handling significant traffic, use RotatingFileHandler with appropriate size limits (10-50 MB) and backup counts (5-10 files) to balance between log retention and disk space.
\end{notebox}

\subsection{TimedRotatingFileHandler}

The \textbf{TimedRotatingFileHandler} rotates log files at specified time intervals rather than by file size.

\subsubsection{Usage Example}

\begin{lstlisting}
from logging.handlers import TimedRotatingFileHandler
import logging

logger = logging.getLogger(__name__)
logger.setLevel(logging.INFO)

# Rotate daily at midnight, keep 30 days of logs
timed_handler = TimedRotatingFileHandler(
    'app.log',
    when='midnight',
    interval=1,
    backupCount=30
)

formatter = logging.Formatter(
    '%(asctime)s - %(levelname)s - %(message)s'
)
timed_handler.setFormatter(formatter)
logger.addHandler(timed_handler)

logger.info("Logging with time-based rotation")
\end{lstlisting}

\subsubsection{Common 'when' Values}

\begin{center}
\begin{tabular}{|l|l|l|}
\hline
\textbf{Value} & \textbf{Type} & \textbf{Description} \\
\hline
'S' & Seconds & Rotate every N seconds \\
\hline
'M' & Minutes & Rotate every N minutes \\
\hline
'H' & Hours & Rotate every N hours \\
\hline
'D' & Days & Rotate every N days \\
\hline
'midnight' & Special & Roll over at midnight \\
\hline
'W0'-'W6' & Weekday & Rotate on specific weekday (0=Monday) \\
\hline
\end{tabular}
\end{center}

\begin{examplebox}{Time-Based Rotation Examples}
\begin{lstlisting}
# Rotate every hour
handler = TimedRotatingFileHandler('app.log', when='H', interval=1)

# Rotate every 6 hours
handler = TimedRotatingFileHandler('app.log', when='H', interval=6)

# Rotate daily at midnight
handler = TimedRotatingFileHandler('app.log', when='midnight')

# Rotate every Monday
handler = TimedRotatingFileHandler('app.log', when='W0')
\end{lstlisting}
\end{examplebox}

\subsection{Multiple Handlers}

A single logger can have multiple handlers, each with different levels, formatters, and destinations. This is extremely useful for production systems.

\begin{examplebox}{Complete Multi-Handler Setup}
\begin{lstlisting}
import logging
from logging.handlers import RotatingFileHandler

# Create logger
logger = logging.getLogger(__name__)
logger.setLevel(logging.DEBUG)  # Logger processes all levels

# Console handler - INFO and above to console
console_handler = logging.StreamHandler()
console_handler.setLevel(logging.INFO)
console_formatter = logging.Formatter(
    '%(levelname)s - %(message)s'
)
console_handler.setFormatter(console_formatter)

# File handler - ERROR and above to file
file_handler = RotatingFileHandler(
    'errors.log',
    maxBytes=5*1024*1024,  # 5MB
    backupCount=3
)
file_handler.setLevel(logging.ERROR)
file_formatter = logging.Formatter(
    '%(asctime)s - %(name)s - %(levelname)s - '
    '[%(filename)s:%(lineno)d] - %(message)s'
)
file_handler.setFormatter(file_formatter)

# Debug file handler - all messages
debug_handler = RotatingFileHandler(
    'debug.log',
    maxBytes=10*1024*1024,  # 10MB
    backupCount=5
)
debug_handler.setLevel(logging.DEBUG)
debug_handler.setFormatter(file_formatter)

# Add all handlers to logger
logger.addHandler(console_handler)
logger.addHandler(file_handler)
logger.addHandler(debug_handler)

# Test different levels
logger.debug("Debug message")      # Only in debug.log
logger.info("Info message")        # Console + debug.log
logger.error("Error message")      # Console + errors.log + debug.log
\end{lstlisting}
\end{examplebox}

This setup provides:
\begin{itemize}[leftmargin=*]
    \item Quick feedback in console (INFO and above)
    \item Separate error log for critical issues
    \item Complete debug log for thorough investigation
\end{itemize}

\newpage

% ========================
% SECTION 5: FORMATTERS
% ========================
\section{Formatters and Message Formatting}

\subsection{Creating Formatters}

Formatters control the final output format of log messages. They can include various pieces of information about the log record.

\subsubsection{Basic Formatter}

\begin{lstlisting}
import logging

# Simple formatter with common fields
formatter = logging.Formatter(
    '%(asctime)s - %(name)s - %(levelname)s - %(message)s'
)

# Apply to handler
handler = logging.StreamHandler()
handler.setFormatter(formatter)
\end{lstlisting}

\subsubsection{Custom Date Format}

\begin{lstlisting}
formatter = logging.Formatter(
    fmt='%(asctime)s - %(levelname)s - %(message)s',
    datefmt='%Y-%m-%d %H:%M:%S'  # Custom date format
)

# Example output: 2024-12-17 14:30:45 - INFO - Message
\end{lstlisting}

\subsection{Common Format Patterns}

\subsubsection{Simple Format (Development)}

Best for quick development and debugging:

\begin{lstlisting}
formatter = logging.Formatter('%(levelname)s - %(message)s')
\end{lstlisting}

Output example:
\begin{verbatim}
INFO - Model training started
ERROR - Failed to load data
\end{verbatim}

\subsubsection{Standard Format (General Purpose)}

Balanced format with essential information:

\begin{lstlisting}
formatter = logging.Formatter(
    '%(asctime)s - %(name)s - %(levelname)s - %(message)s'
)
\end{lstlisting}

Output example:
\begin{verbatim}
2024-12-17 14:30:45,123 - ml_model - INFO - Model training started
\end{verbatim}

\subsubsection{Detailed Format (Production)}

Includes file and line information for debugging:

\begin{lstlisting}
formatter = logging.Formatter(
    '%(asctime)s - %(name)s - %(levelname)s - '
    '[%(filename)s:%(lineno)d] - %(message)s',
    datefmt='%Y-%m-%d %H:%M:%S'
)
\end{lstlisting}

Output example:
\begin{verbatim}
2024-12-17 14:30:45 - ml_model - INFO - [train.py:42] - Model training started
\end{verbatim}

\subsubsection{MLOps Format (Comprehensive)}

Includes process/thread info for distributed systems:

\begin{lstlisting}
formatter = logging.Formatter(
    
    '%(asctime)s - [PID:%(process)d TID:%(thread)d] - '
    '%(name)s - %(funcName)s - %(levelname)s - %(message)s',
    datefmt='%Y-%m-%d %H:%M:%S'
)
\end{lstlisting}

Output example:
\begin{verbatim}
2024-12-17 14:30:45 - [PID:12345 TID:67890] - ml_model - 
train_model - INFO - Epoch 1 complete
\end{verbatim}

\subsection{Complete Format Attributes Reference}

\begin{center}
\begin{tabular}{|p{0.28\textwidth}|p{0.62\textwidth}|}
\hline
\textbf{Attribute} & \textbf{Description} \\
\hline
\texttt{\%(name)s} & Logger name (typically module name) \\
\hline
\texttt{\%(levelname)s} & Text logging level (DEBUG, INFO, etc.) \\
\hline
\texttt{\%(levelno)s} & Numeric logging level (10, 20, etc.) \\
\hline
\texttt{\%(pathname)s} & Full pathname of source file \\
\hline
\texttt{\%(filename)s} & Filename portion of pathname \\
\hline
\texttt{\%(module)s} & Module name (filename without .py) \\
\hline
\texttt{\%(lineno)d} & Line number where log was called \\
\hline
\texttt{\%(funcName)s} & Function name where log was called \\
\hline
\texttt{\%(created)f} & Time when LogRecord was created (seconds since epoch) \\
\hline
\texttt{\%(asctime)s} & Human-readable time \\
\hline
\texttt{\%(msecs)d} & Millisecond portion of time \\
\hline
\texttt{\%(relativeCreated)d} & Time in ms since logging module loaded \\
\hline
\texttt{\%(thread)d} & Thread ID \\
\hline
\texttt{\%(threadName)s} & Thread name \\
\hline
\texttt{\%(process)d} & Process ID \\
\hline
\texttt{\%(message)s} & The logged message \\
\hline
\end{tabular}
\end{center}

\newpage

% ========================
% SECTION 6: CONFIGURATION
% ========================
\section{Logging Configuration}

\subsection{Basic Configuration with basicConfig()}

The \texttt{basicConfig()} function provides a quick way to configure logging for simple applications.

\subsubsection{Simple Setup}

\begin{lstlisting}
import logging

# Basic configuration with default settings
logging.basicConfig(level=logging.INFO)

# Use root logger
logging.info("This is an info message")
logging.error("This is an error message")
\end{lstlisting}

\subsubsection{Detailed Configuration}

\begin{lstlisting}
import logging

logging.basicConfig(
    level=logging.DEBUG,
    format='%(asctime)s - %(name)s - %(levelname)s - %(message)s',
    datefmt='%Y-%m-%d %H:%M:%S',
    filename='app.log',
    filemode='w'  # 'w' to overwrite, 'a' to append
)

logging.info("Application started")
logging.debug("Debug information")
\end{lstlisting}

\subsubsection{Multiple Handlers with basicConfig()}

\begin{lstlisting}
import logging

logging.basicConfig(
    level=logging.DEBUG,
    format='%(asctime)s - %(levelname)s - %(message)s',
    handlers=[
        logging.FileHandler('debug.log'),
        logging.StreamHandler()
    ]
)

logging.info("Logs to both file and console")
\end{lstlisting}

\begin{warningbox}
\textbf{Important Limitation}: \texttt{basicConfig()} only works the first time it's called. Subsequent calls are ignored unless you use the \texttt{force=True} parameter (Python 3.8+).

\begin{lstlisting}
# Python 3.8+ only
logging.basicConfig(level=logging.DEBUG, force=True)
\end{lstlisting}
\end{warningbox}

\subsection{Manual Configuration (Recommended)}

For production applications, manual configuration provides more control and flexibility.

\subsubsection{Complete Setup Example}

\begin{lstlisting}
import logging
from logging.handlers import RotatingFileHandler

def setup_logger(name, log_file, level=logging.INFO):
    """
    Function to setup logger with console and file handlers.
    
    Args:
        name: Logger name
        log_file: Path to log file
        level: Logging level
    
    Returns:
        Configured logger object
    """
    
    # Create logger
    logger = logging.getLogger(name)
    logger.setLevel(level)
    
    # Prevent duplicate handlers
    if logger.handlers:
        return logger
    
    # Create formatters
    console_formatter = logging.Formatter(
        '%(levelname)s - %(message)s'
    )
    file_formatter = logging.Formatter(
        '%(asctime)s - %(name)s - %(levelname)s - '
        '[%(filename)s:%(lineno)d] - %(message)s',
        datefmt='%Y-%m-%d %H:%M:%S'
    )
    
    # Console handler
    console_handler = logging.StreamHandler()
    console_handler.setLevel(logging.INFO)
    console_handler.setFormatter(console_formatter)
    
    # File handler with rotation
    file_handler = RotatingFileHandler(
        log_file,
        maxBytes=10*1024*1024,  # 10MB
        backupCount=5
    )
    file_handler.setLevel(logging.DEBUG)
    file_handler.setFormatter(file_formatter)
    
    # Add handlers to logger
    logger.addHandler(console_handler)
    logger.addHandler(file_handler)
    
    return logger

# Usage
logger = setup_logger('ml_pipeline', 'pipeline.log')
logger.info("Logger configured successfully")
logger.debug("This goes to file only")
\end{lstlisting}

\subsection{Configuration Using Dictionary}

For complex applications, dictionary-based configuration provides a clean, declarative approach.

\begin{lstlisting}
import logging
import logging.config

LOGGING_CONFIG = {
    'version': 1,
    'disable_existing_loggers': False,
    
    'formatters': {
        'standard': {
            'format': '%(asctime)s - %(name)s - %(levelname)s - %(message)s'
        },
        'detailed': {
            'format': '%(asctime)s - %(name)s - %(levelname)s - '
                     '[%(filename)s:%(lineno)d] - %(message)s'
        }
    },
    
    'handlers': {
        'console': {
            'class': 'logging.StreamHandler',
            'level': 'INFO',
            'formatter': 'standard',
            'stream': 'ext://sys.stdout'
        },
        'file': {
            'class': 'logging.handlers.RotatingFileHandler',
            'level': 'DEBUG',
            'formatter': 'detailed',
            'filename': 'app.log',
            'maxBytes': 10485760,  # 10MB
            'backupCount': 5
        },
        'error_file': {
            'class': 'logging.FileHandler',
            'level': 'ERROR',
            'formatter': 'detailed',
            'filename': 'errors.log'
        }
    },
    
    'loggers': {
        '': {  # Root logger
            'handlers': ['console', 'file', 'error_file'],
            'level': 'DEBUG',
            'propagate': False
        },
        'ml_model': {  # Specific logger
            'handlers': ['console', 'file'],
            'level': 'INFO',
            'propagate': False
        }
    }
}

# Apply configuration
logging.config.dictConfig(LOGGING_CONFIG)

# Use loggers
root_logger = logging.getLogger()
ml_logger = logging.getLogger('ml_model')

root_logger.info("Root logger message")
ml_logger.info("ML model logger message")
\end{lstlisting}

\subsection{Configuration from File}

You can store configuration in external files for easy management.

\subsubsection{YAML Configuration File}

Create \texttt{logging\_config.yaml}:

\begin{lstlisting}
version: 1
disable_existing_loggers: False

formatters:
  standard:
    format: '%(asctime)s - %(name)s - %(levelname)s - %(message)s'

handlers:
  console:
    class: logging.StreamHandler
    level: INFO
    formatter: standard
    stream: ext://sys.stdout
  
  file:
    class: logging.handlers.RotatingFileHandler
    level: DEBUG
    formatter: standard
    filename: app.log
    maxBytes: 10485760
    backupCount: 5

loggers:
  '':
    handlers: [console, file]
    level: DEBUG
    propagate: False
\end{lstlisting}

Load and apply:

\begin{lstlisting}
import logging.config
import yaml

with open('logging_config.yaml', 'r') as f:
    config = yaml.safe_load(f)
    logging.config.dictConfig(config)

logger = logging.getLogger(__name__)
logger.info("Configuration loaded from YAML")
\end{lstlisting}

\newpage

% ========================
% SECTION 7: BEST PRACTICES
% ========================
\section{Best Practices for MLOps}

\subsection{General Logging Best Practices}

\begin{enumerate}[leftmargin=*]
    \item \textbf{Use \_\_name\_\_ for logger names}
    
    This creates a hierarchical logger structure based on your module organization:
    
    \begin{lstlisting}
# Good practice
logger = logging.getLogger(__name__)

# Avoid
logger = logging.getLogger('my_logger')
    \end{lstlisting}
    
    \item \textbf{Use appropriate logging levels}
    
    \begin{itemize}
        \item DEBUG: Development and debugging only
        \item INFO: Production tracking and milestones
        \item WARNING: Recoverable issues and deprecations
        \item ERROR: Failures that affect functionality
        \item CRITICAL: System-level failures
    \end{itemize}
    
    \item \textbf{Use lazy formatting for performance}
    
    \begin{lstlisting}
# Good - lazy evaluation (only formats if logged)
logger.info("Processing batch %d with size %d", batch_id, size)

# Bad - eager evaluation (always formats, even if not logged)
logger.info(f"Processing batch {batch_id} with size {size}")

# Also bad
logger.info("Processing batch " + str(batch_id) + " with size " + str(size))
    \end{lstlisting}
    
    \item \textbf{Use logger.exception() in except blocks}
    
    \begin{lstlisting}
try:
    model.predict(data)
except Exception as e:
    logger.exception("Prediction failed")
    # Automatically includes full traceback
    raise  # Re-raise after logging
    \end{lstlisting}
    
    \item \textbf{Don't log sensitive information}
    
    \begin{lstlisting}
# Bad - security risk
logger.info(f"User password: {password}")
logger.info(f"API key: {api_key}")

# Good - no sensitive data
logger.info(f"User authentication attempt for: {username}")
logger.info("API authentication successful")
    \end{lstlisting}
    
    \item \textbf{Configure logging once at startup}
    
    Set up logging configuration at application entry point, not in every module.
    
    \item \textbf{Use structured logging for complex data}
    
    \begin{lstlisting}
# Use extra parameter for structured data
logger.info(
    "Model evaluation complete",
    extra={
        'accuracy': 0.95,
        'precision': 0.93,
        'recall': 0.97,
        'f1_score': 0.95
    }
)
    \end{lstlisting}
    
    \item \textbf{Avoid excessive logging}
    
    \begin{lstlisting}
# Bad - logs every iteration (too verbose)
for i in range(10000):
    logger.debug(f"Processing item {i}")

# Good - log periodically
for i in range(10000):
    if i % 1000 == 0:
        logger.info(f"Processed {i}/10000 items")
    \end{lstlisting}
\end{enumerate}

\subsection{MLOps-Specific Best Practices}

\subsubsection{Log Model Training Progress}

\begin{lstlisting}
import logging

logger = logging.getLogger(__name__)

def train_model(model, train_loader, val_loader, epochs):
    logger.info("=" * 60)
    logger.info("STARTING MODEL TRAINING")
    logger.info("=" * 60)
    logger.info(f"Training samples: {len(train_loader.dataset)}")
    logger.info(f"Validation samples: {len(val_loader.dataset)}")
    logger.info(f"Epochs: {epochs}")
    
    for epoch in range(epochs):
        # Training phase
        train_loss = train_epoch(model, train_loader)
        
        # Validation phase
        val_loss, val_acc = validate(model, val_loader)
        
        # Log progress
        logger.info(
            f"Epoch {epoch+1}/{epochs}: "
            f"train_loss={train_loss:.4f}, "
            f"val_loss={val_loss:.4f}, "
            f"val_acc={val_acc:.4f}"
        )
        
        # Log warnings if needed
        if val_loss > train_loss * 1.5:
            logger.warning("Possible overfitting detected")
    
    logger.info("Training completed successfully")
    logger.info(f"Final validation accuracy: {val_acc:.4f}")
    logger.info("=" * 60)
\end{lstlisting}

\subsubsection{Log Data Pipeline Steps}

\begin{lstlisting}
def process_data_pipeline(data_source):
    logger.info(f"Loading data from {data_source}")
    data = load_data(data_source)
    logger.debug(f"Raw data shape: {data.shape}")
    
    logger.info("Starting data preprocessing")
    missing_count = data.isnull().sum().sum()
    if missing_count > 0:
        logger.warning(f"Found {missing_count} missing values")
    
    cleaned_data = preprocess(data)
    logger.debug(f"Cleaned data shape: {cleaned_data.shape}")
    
    logger.info("Feature engineering started")
    features = engineer_features(cleaned_data)
    logger.info(f"Created {features.shape[1]} features")
    logger.debug(f"Feature names: {list(features.columns)}")
    
    logger.info("Data pipeline completed successfully")
    logger.info(f"Final dataset shape: {features.shape}")
    
    return features
\end{lstlisting}

\subsubsection{Log Model Predictions with Timing}

\begin{lstlisting}
import time

def predict_with_logging(model, input_data, request_id=None):
    if request_id is None:
        request_id = str(time.time())
    
    logger.info(f"[{request_id}] Prediction request received")
    logger.debug(f"[{request_id}] Input shape: {input_data.shape}")
    
    try:
        start_time = time.time()
        predictions = model.predict(input_data)
        inference_time = time.time() - start_time
        
        logger.info(
            f"[{request_id}] Prediction completed in "
            f"{inference_time:.3f}s"
        )
        logger.debug(
            f"[{request_id}] Predictions: {predictions[:5]}..."
        )  # Log first 5 only
        
        return predictions
        
    except Exception as e:
        logger.exception(f"[{request_id}] Prediction failed")
        raise
\end{lstlisting}

\subsubsection{Log Experiment Configuration}

\begin{lstlisting}
def log_experiment_config(config):
    logger.info("=" * 70)
    logger.info("EXPERIMENT CONFIGURATION")
    logger.info("=" * 70)
    
    for section, params in config.items():
        logger.info(f"{section}:")
        if isinstance(params, dict):
            for key, value in params.items():
                logger.info(f"  {key}: {value}")
        else:
            logger.info(f"  {params}")
    
    logger.info("=" * 70)

# Usage
config = {
    'model': {
        'type': 'RandomForest',
        'n_estimators': 100,
        'max_depth': 10
    },
    'training': {
        'epochs': 50,
        'batch_size': 32,
        'learning_rate': 0.001
    },
    'data': {
        'train_split': 0.8,
        'random_state': 42
    }
}

log_experiment_config(config)
\end{lstlisting}

\subsection{Error Handling Best Practices}

\begin{lstlisting}
def load_and_process_data(file_path):
    try:
        logger.info(f"Attempting to load data from {file_path}")
        data = pd.read_csv(file_path)
        
        logger.info(f"Data loaded successfully: {data.shape}")
        logger.debug(f"Columns: {list(data.columns)}")
        
        # Data validation
        if data.empty:
            logger.error("Loaded data is empty")
            raise ValueError("Empty dataset")
        
        # Processing
        logger.info("Starting data processing")
        processed_data = process(data)
        
        logger.info("Data processing completed successfully")
        return processed_data
        
    except FileNotFoundError:
        logger.error(f"File not found: {file_path}")
        logger.error("Please check the file path and try again")
        raise
        
    except pd.errors.EmptyDataError:
        logger.error(f"Empty data file: {file_path}")
        raise
        
    except pd.errors.ParserError as e:
        logger.error(f"Error parsing CSV file: {str(e)}")
        logger.error("Check file format and encoding")
        raise
        
    except MemoryError:
        logger.critical(
            f"Out of memory while loading {file_path}"
        )
        logger.critical("Try loading data in chunks")
        raise
        
    except Exception as e:
        logger.exception(
            f"Unexpected error loading data from {file_path}"
        )
        raise
\end{lstlisting}

\newpage

% ========================
% SECTION 8: REAL-WORLD EXAMPLES
% ========================
\section{Complete Real-World Examples}

\subsection{ML Training Script with Comprehensive Logging}

\begin{lstlisting}
import logging
from logging.handlers import RotatingFileHandler
import time
import numpy as np
from sklearn.model_selection import train_test_split
from sklearn.ensemble import RandomForestClassifier
from sklearn.metrics import (
    accuracy_score, precision_score, 
    recall_score, f1_score
)

# ============================================
# Logger Configuration
# ============================================

def setup_training_logger():
    """Setup logger for ML training with file rotation"""
    logger = logging.getLogger(__name__)
    logger.setLevel(logging.DEBUG)
    
    # Console handler - INFO and above
    console_handler = logging.StreamHandler()
    console_handler.setLevel(logging.INFO)
    console_formatter = logging.Formatter(
        '%(levelname)s - %(message)s'
    )
    console_handler.setFormatter(console_formatter)
    
    # File handler - all messages with rotation
    file_handler = RotatingFileHandler(
        'training.log',
        maxBytes=10*1024*1024,  # 10MB
        backupCount=5
    )
    file_handler.setLevel(logging.DEBUG)
    file_formatter = logging.Formatter(
        '%(asctime)s - %(name)s - %(levelname)s - '
        '[%(filename)s:%(lineno)d] - %(message)s',
        datefmt='%Y-%m-%d %H:%M:%S'
    )
    file_handler.setFormatter(file_formatter)
    
    # Add handlers
    logger.addHandler(console_handler)
    logger.addHandler(file_handler)
    
    return logger

logger = setup_training_logger()

# ============================================
# Training Functions
# ============================================

def validate_data(X, y):
    """Validate input data with logging"""
    logger.debug("Validating input data")
    
    if X is None or y is None:
        logger.error("Input data is None")
        raise ValueError("Invalid input data")
    
    if len(X) != len(y):
        logger.error(
            f"Shape mismatch: X has {len(X)} samples, "
            f"y has {len(y)} samples"
        )
        raise ValueError("X and y length mismatch")
    
    if len(X) < 100:
        logger.warning(
            f"Small dataset: only {len(X)} samples"
        )
    
    logger.debug(f"Data validation passed: {len(X)} samples")

def train_model(X, y, config):
    """
    Train ML model with comprehensive logging
    
    Args:
        X: Feature matrix
        y: Target vector
        config: Training configuration dictionary
    
    Returns:
        Trained model
    """
    
    logger.info("=" * 70)
    logger.info("STARTING MODEL TRAINING")
    logger.info("=" * 70)
    
    # Log configuration
    logger.info("Training Configuration:")
    for key, value in config.items():
        logger.info(f"  {key}: {value}")
    logger.info("-" * 70)
    
    try:
        # Validate data
        validate_data(X, y)
        
        # Split data
        logger.info("Splitting data into train and test sets")
        X_train, X_test, y_train, y_test = train_test_split(
            X, y, 
            test_size=config['test_size'],
            random_state=config['random_state']
        )
        
        logger.info(f"Training set: {len(X_train)} samples")
        logger.info(f"Test set: {len(X_test)} samples")
        logger.debug(f"Training features shape: {X_train.shape}")
        logger.debug(f"Test features shape: {X_test.shape}")
        
        # Check class balance
        unique, counts = np.unique(y_train, return_counts=True)
        logger.debug(f"Class distribution: {dict(zip(unique, counts))}")
        if max(counts) / min(counts) > 10:
            logger.warning("Highly imbalanced dataset detected")
        
        # Initialize model
        logger.info("Initializing Random Forest model")
        logger.debug(f"Model parameters: {config}")
        
        model = RandomForestClassifier(
            n_estimators=config['n_estimators'],
            max_depth=config['max_depth'],
            random_state=config['random_state'],
            n_jobs=-1,  # Use all CPU cores
            verbose=0
        )
        
        # Train model
        logger.info("Starting model training...")
        start_time = time.time()
        
        model.fit(X_train, y_train)
        
        training_time = time.time() - start_time
        logger.info(
            f"Training completed in {training_time:.2f} seconds"
        )
        
        # Evaluate on training set
        logger.debug("Evaluating on training set")
        y_train_pred = model.predict(X_train)
        train_acc = accuracy_score(y_train, y_train_pred)
        logger.debug(f"Training accuracy: {train_acc:.4f}")
        
        # Evaluate on test set
        logger.info("Evaluating model on test set")
        start_eval = time.time()
        y_pred = model.predict(X_test)
        eval_time = time.time() - start_eval
        
        logger.debug(f"Evaluation time: {eval_time:.3f}s")
        
        # Calculate metrics
        accuracy = accuracy_score(y_test, y_pred)
        precision = precision_score(
            y_test, y_pred, average='weighted', zero_division=0
        )
        recall = recall_score(
            y_test, y_pred, average='weighted', zero_division=0
        )
        f1 = f1_score(
            y_test, y_pred, average='weighted', zero_division=0
        )
        
        # Log results
        logger.info("=" * 70)
        logger.info("MODEL EVALUATION RESULTS")
        logger.info("=" * 70)
        logger.info(f"Accuracy:  {accuracy:.4f}")
        logger.info(f"Precision: {precision:.4f}")
        logger.info(f"Recall:    {recall:.4f}")
        logger.info(f"F1-Score:  {f1:.4f}")
        logger.info("=" * 70)
        
        # Feature importance
        if hasattr(model, 'feature_importances_'):
            importance = model.feature_importances_
            logger.debug(f"Feature importances: {importance}")
            top_features = np.argsort(importance)[-5:][::-1]
            logger.info(
                f"Top 5 important features: {top_features.tolist()}"
            )
        
        # Performance warnings
        if accuracy < 0.7:
            logger.warning(
                f"Low accuracy: {accuracy:.4f}. "
                "Consider feature engineering or hyperparameter tuning"
            )
        
        if training_time > 300:  # 5 minutes
            logger.warning(
                f"Long training time: {training_time:.2f}s. "
                "Consider reducing model complexity"
            )
        
        logger.info("Model training pipeline completed successfully")
        
        return model
        
    except ValueError as e:
        logger.error(f"Value error during training: {str(e)}")
        raise
        
    except MemoryError:
        logger.critical(
            "Out of memory during model training. "
            "Try reducing dataset size or model complexity"
        )
        raise
        
    except Exception as e:
        logger.exception("Unexpected error during model training")
        raise

# ============================================
# Main Execution
# ============================================

if __name__ == "__main__":
    logger.info("Script started")
    logger.info(f"Python logging version: {logging.__version__}")
    
    # Configuration
    config = {
        'n_estimators': 100,
        'max_depth': 10,
        'test_size': 0.2,
        'random_state': 42
    }
    
    try:
        # Generate sample data
        logger.info("Generating sample dataset")
        np.random.seed(42)
        X = np.random.rand(1000, 10)
        y = np.random.randint(0, 2, 1000)
        logger.info("Sample data generated successfully")
        
        # Train model
        model = train_model(X, y, config)
        
        logger.info("Script completed successfully")
        
    except KeyboardInterrupt:
        logger.warning("Script interrupted by user")
    except Exception as e:
        logger.critical("Script failed with critical error")
        logger.exception("Error details")
        raise
\end{lstlisting}

\subsection{Data Pipeline with Logging}

\begin{lstlisting}
import logging
import pandas as pd
import numpy as np
from datetime import datetime

logger = logging.getLogger(__name__)

class DataPipeline:
    """
    Data pipeline with comprehensive logging
    
    Handles data loading, cleaning, and feature engineering
    with detailed logging at each step.
    """
    
    def __init__(self, config):
        self.config = config
        logger.info("DataPipeline initialized")
        logger.debug(f"Pipeline config: {config}")
    
    def load_data(self, file_path):
        """Load data with error handling and logging"""
        try:
            logger.info(f"Loading data from: {file_path}")
            start_time = datetime.now()
            
            data = pd.read_csv(file_path)
            
            load_time = (datetime.now() - start_time).total_seconds()
            logger.info(f"Data loaded in {load_time:.2f} seconds")
            logger.info(f"Dataset shape: {data.shape}")
            logger.debug(f"Columns ({len(data.columns)}): {list(data.columns)}")
            logger.debug(f"Memory usage: {data.memory_usage(deep=True).sum() / 1024**2:.2f} MB")
            
            # Log data types
            logger.debug("Data types:")
            for col, dtype in data.dtypes.items():
                logger.debug(f"  {col}: {dtype}")
            
            return data
            
        except FileNotFoundError:
            logger.error(f"File not found: {file_path}")
            logger.error("Please verify the file path")
            raise
            
        except pd.errors.ParserError as e:
            logger.error(f"Error parsing CSV: {str(e)}")
            logger.error("Check file format, delimiter, and encoding")
            raise
            
        except pd.errors.EmptyDataError:
            logger.error(f"Empty file: {file_path}")
            raise
            
        except Exception as e:
            logger.exception("Unexpected error loading data")
            raise
    
    def clean_data(self, data):
        """Clean data with detailed logging"""
        logger.info("Starting data cleaning")
        
        original_shape = data.shape
        original_rows = original_shape[0]
        logger.debug(f"Original shape: {original_shape}")
        
        # Check missing values
        missing_count = data.isnull().sum().sum()
        if missing_count > 0:
            missing_pct = (missing_count / data.size) * 100
            logger.warning(
                f"Found {missing_count} missing values "
                f"({missing_pct:.2f}% of data)"
            )
            
            # Log missing values per column
            missing_cols = data.isnull().sum()
            missing_cols = missing_cols[missing_cols > 0]
            for col, count in missing_cols.items():
                col_pct = (count / len(data)) * 100
                logger.debug(
                    f"  {col}: {count} missing ({col_pct:.1f}%)"
                )
            
            # Handle missing values
            logger.info("Handling missing values")
            numeric_cols = data.select_dtypes(
                include=[np.number]
            ).columns
            data[numeric_cols] = data[numeric_cols].fillna(
                data[numeric_cols].mean()
            )
            logger.info("Numeric missing values filled with mean")
            
                        categorical_cols = data.select_dtypes(
                include=['object']
            ).columns
            data[categorical_cols] = data[categorical_cols].fillna(
                data[categorical_cols].mode().iloc[0]
            )
            logger.info("Categorical missing values filled with mode")
        else:
            logger.info("No missing values found")
        
        # Remove duplicates
        duplicates = data.duplicated().sum()
        if duplicates > 0:
            dup_pct = (duplicates / len(data)) * 100
            logger.warning(
                f"Found {duplicates} duplicate rows ({dup_pct:.2f}%)"
            )
            data = data.drop_duplicates()
            logger.info(f"Removed {duplicates} duplicate rows")
        else:
            logger.info("No duplicates found")
        
        # Remove outliers (optional)
        numeric_cols = data.select_dtypes(include=[np.number]).columns
        outliers_removed = 0
        for col in numeric_cols:
            Q1 = data[col].quantile(0.25)
            Q3 = data[col].quantile(0.75)
            IQR = Q3 - Q1
            lower_bound = Q1 - 1.5 * IQR
            upper_bound = Q3 + 1.5 * IQR
            
            outliers = ((data[col] < lower_bound) | 
                       (data[col] > upper_bound)).sum()
            if outliers > 0:
                logger.debug(f"  {col}: {outliers} outliers detected")
                outliers_removed += outliers
        
        if outliers_removed > 0:
            logger.info(f"Total outliers detected: {outliers_removed}")
        
        final_shape = data.shape
        rows_removed = original_rows - final_shape[0]
        
        logger.info("Data cleaning completed")
        logger.info(f"Final shape: {final_shape}")
        logger.info(f"Rows removed: {rows_removed}")
        
        return data
    
    def feature_engineering(self, data):
        """Perform feature engineering with logging"""
        logger.info("Starting feature engineering")
        
        original_features = data.shape[1]
        logger.debug(f"Original features: {original_features}")
        
        try:
            # Example: Create interaction features
            logger.debug("Creating derived features")
            
            # Add timestamp features if date column exists
            date_cols = data.select_dtypes(
                include=['datetime64']
            ).columns
            if len(date_cols) > 0:
                logger.info(
                    f"Processing {len(date_cols)} datetime columns"
                )
                for col in date_cols:
                    data[f'{col}_year'] = data[col].dt.year
                    data[f'{col}_month'] = data[col].dt.month
                    data[f'{col}_day'] = data[col].dt.day
                    logger.debug(
                        f"Created time features from {col}"
                    )
            
            new_features = data.shape[1]
            added_features = new_features - original_features
            
            logger.info("Feature engineering completed")
            logger.info(f"Total features: {new_features}")
            logger.info(f"Added features: {added_features}")
            logger.debug(f"New columns: {list(data.columns)}")
            
            return data
            
        except Exception as e:
            logger.exception("Error during feature engineering")
            raise
    
    def run(self, file_path):
        """Execute complete pipeline"""
        logger.info("=" * 70)
        logger.info("STARTING DATA PIPELINE")
        logger.info("=" * 70)
        
        pipeline_start = datetime.now()
        
        try:
            # Load data
            data = self.load_data(file_path)
            
            # Clean data
            data = self.clean_data(data)
            
            # Feature engineering
            data = self.feature_engineering(data)
            
            pipeline_time = (
                datetime.now() - pipeline_start
            ).total_seconds()
            
            logger.info("=" * 70)
            logger.info("PIPELINE COMPLETED SUCCESSFULLY")
            logger.info(
                f"Total pipeline time: {pipeline_time:.2f} seconds"
            )
            logger.info("=" * 70)
            
            return data
            
        except Exception as e:
            logger.critical("Pipeline failed")
            logger.exception("Pipeline error details")
            raise

# Usage example
if __name__ == "__main__":
    logging.basicConfig(
        level=logging.DEBUG,
        format='%(asctime)s - %(name)s - %(levelname)s - %(message)s',
        handlers=[
            logging.FileHandler('pipeline.log'),
            logging.StreamHandler()
        ]
    )
    
    config = {'version': '1.0', 'mode': 'production'}
    pipeline = DataPipeline(config)
    
    # Run pipeline
    try:
        result = pipeline.run('data.csv')
        logger.info("Pipeline execution successful")
    except Exception:
        logger.error("Pipeline execution failed")
\end{lstlisting}

\subsection{Model Serving API with Logging}

\begin{lstlisting}
import logging
from flask import Flask, request, jsonify
import numpy as np
import time
import uuid

app = Flask(__name__)

# Configure logging
logging.basicConfig(
    level=logging.INFO,
    format='%(asctime)s - %(name)s - %(levelname)s - %(message)s',
    handlers=[
        logging.FileHandler('api.log'),
        logging.StreamHandler()
    ]
)

logger = logging.getLogger(__name__)

# Global model variable
model = None

@app.before_first_request
def setup():
    """Initialize application"""
    global model
    logger.info("=" * 70)
    logger.info("INITIALIZING MODEL SERVING API")
    logger.info("=" * 70)
    
    try:
        logger.info("Loading model")
        # model = load_model('model.pkl')  # Placeholder
        logger.info("Model loaded successfully")
        logger.info("API initialization complete")
    except Exception as e:
        logger.critical("Failed to load model")
        logger.exception("Model loading error")
        raise

@app.route('/predict', methods=['POST'])
def predict():
    """Prediction endpoint with comprehensive logging"""
    request_id = str(uuid.uuid4())[:8]
    
    logger.info(f"[{request_id}] Prediction request received")
    logger.debug(f"[{request_id}] Request headers: {dict(request.headers)}")
    
    try:
        # Parse request
        data = request.get_json()
        logger.debug(f"[{request_id}] Request data keys: {list(data.keys())}")
        
        # Validate request
        if not data or 'features' not in data:
            logger.warning(
                f"[{request_id}] Invalid request: missing 'features'"
            )
            return jsonify({'error': 'Missing features field'}), 400
        
        features = np.array(data['features'])
        logger.debug(f"[{request_id}] Features shape: {features.shape}")
        
        # Validate input shape
        if features.ndim != 2:
            logger.error(
                f"[{request_id}] Invalid shape: {features.shape}. "
                f"Expected 2D array"
            )
            return jsonify({
                'error': f'Invalid feature shape: {features.shape}'
            }), 400
        
        # Check for invalid values
        if np.isnan(features).any():
            logger.error(f"[{request_id}] NaN values in input")
            return jsonify({'error': 'NaN values in features'}), 400
        
        if np.isinf(features).any():
            logger.error(f"[{request_id}] Inf values in input")
            return jsonify({'error': 'Inf values in features'}), 400
        
        # Make prediction
        logger.info(f"[{request_id}] Starting prediction")
        start_time = time.time()
        
        # prediction = model.predict(features)  # Placeholder
        prediction = np.random.rand(features.shape[0])  # Mock prediction
        
        inference_time = time.time() - start_time
        
        logger.info(
            f"[{request_id}] Prediction completed in "
            f"{inference_time:.3f}s"
        )
        logger.debug(
            f"[{request_id}] Prediction shape: {prediction.shape}"
        )
        logger.debug(
            f"[{request_id}] Sample predictions: {prediction[:3]}"
        )
        
        # Log slow predictions
        if inference_time > 1.0:
            logger.warning(
                f"[{request_id}] Slow prediction: {inference_time:.3f}s"
            )
        
        response = {
            'request_id': request_id,
            'prediction': prediction.tolist(),
            'inference_time': inference_time,
            'status': 'success'
        }
        
        logger.info(f"[{request_id}] Request completed successfully")
        return jsonify(response), 200
        
    except ValueError as e:
        logger.error(f"[{request_id}] Value error: {str(e)}")
        return jsonify({
            'error': 'Invalid input values',
            'details': str(e)
        }), 400
        
    except Exception as e:
        logger.exception(f"[{request_id}] Prediction failed")
        return jsonify({
            'error': 'Internal server error',
            'request_id': request_id
        }), 500

@app.route('/health', methods=['GET'])
def health():
    """Health check endpoint"""
    logger.debug("Health check requested")
    return jsonify({
        'status': 'healthy',
        'timestamp': time.time()
    }), 200

@app.route('/metrics', methods=['GET'])
def metrics():
    """Metrics endpoint"""
    logger.debug("Metrics requested")
    # Return application metrics
    return jsonify({
        'requests_total': 0,  # Placeholder
        'errors_total': 0,    # Placeholder
        'avg_inference_time': 0.0  # Placeholder
    }), 200

if __name__ == '__main__':
    logger.info("Starting Flask application")
    logger.info("Server configuration:")
    logger.info("  Host: 0.0.0.0")
    logger.info("  Port: 5000")
    logger.info("  Debug: False")
    
    app.run(host='0.0.0.0', port=5000, debug=False)
\end{lstlisting}

\newpage

% ========================
% SECTION 9: ADVANCED TOPICS
% ========================
\section{Advanced Logging Topics}

\subsection{Logger Hierarchy}

Loggers follow a hierarchical naming structure using dots (.), similar to Python's module structure.

\begin{lstlisting}
import logging

# Parent logger
parent_logger = logging.getLogger('myapp')

# Child loggers (automatically inherit from parent)
data_logger = logging.getLogger('myapp.data')
model_logger = logging.getLogger('myapp.model')
api_logger = logging.getLogger('myapp.api')

# Grandchild logger
preprocessing_logger = logging.getLogger('myapp.data.preprocessing')
\end{lstlisting}

\textbf{Propagation Rules}:
\begin{itemize}[leftmargin=*]
    \item Child loggers inherit settings from parent loggers
    \item Messages propagate up the hierarchy by default
    \item Set \texttt{propagate=False} to prevent propagation
    \item Root logger is at the top of all hierarchies
\end{itemize}

\begin{lstlisting}
# Configure parent logger
parent = logging.getLogger('myapp')
parent.setLevel(logging.INFO)
handler = logging.StreamHandler()
parent.addHandler(handler)

# Child automatically inherits configuration
child = logging.getLogger('myapp.module')
child.info("This uses parent's handler")  # Works!

# Disable propagation
child.propagate = False  # Now won't use parent's handlers
\end{lstlisting}

\subsection{Custom Log Levels}

You can define custom logging levels for specialized needs:

\begin{lstlisting}
import logging

# Define custom level between DEBUG and INFO
TRACE = 5
logging.addLevelName(TRACE, "TRACE")

def trace(self, message, *args, **kwargs):
    """Add trace method to logger"""
    if self.isEnabledFor(TRACE):
        self._log(TRACE, message, args, **kwargs)

# Add method to Logger class
logging.Logger.trace = trace

# Usage
logger = logging.getLogger(__name__)
logger.setLevel(TRACE)

handler = logging.StreamHandler()
handler.setLevel(TRACE)
logger.addHandler(handler)

logger.trace("This is a trace message")  # Works!
logger.debug("This is a debug message")
\end{lstlisting}

\subsection{Filters}

Filters provide fine-grained control over which log records are processed:

\begin{lstlisting}
import logging

class LevelRangeFilter(logging.Filter):
    """Filter to only allow specific level range"""
    
    def __init__(self, min_level, max_level):
        super().__init__()
        self.min_level = min_level
        self.max_level = max_level
    
    def filter(self, record):
        return self.min_level <= record.levelno <= self.max_level

# Usage
logger = logging.getLogger(__name__)
handler = logging.FileHandler('info_warnings.log')

# Only log INFO and WARNING (not DEBUG, ERROR, CRITICAL)
level_filter = LevelRangeFilter(logging.INFO, logging.WARNING)
handler.addFilter(level_filter)

logger.addHandler(handler)

logger.debug("Not logged")        # Below min
logger.info("Logged")             # In range
logger.warning("Logged")          # In range
logger.error("Not logged")        # Above max
\end{lstlisting}

\subsubsection{Context Filter Example}

\begin{lstlisting}
class ContextFilter(logging.Filter):
    """Add contextual information to logs"""
    
    def __init__(self, user_id=None):
        super().__init__()
        self.user_id = user_id
    
    def filter(self, record):
        record.user_id = self.user_id or 'anonymous'
        return True

# Setup
logger = logging.getLogger(__name__)
handler = logging.StreamHandler()
formatter = logging.Formatter(
    '%(asctime)s - [User:%(user_id)s] - %(levelname)s - %(message)s'
)
handler.setFormatter(formatter)

# Add filter
context_filter = ContextFilter(user_id='user123')
handler.addFilter(context_filter)

logger.addHandler(handler)

logger.info("Processing request")
# Output: 2024-12-17 14:30:45 - [User:user123] - INFO - Processing request
\end{lstlisting}

\subsection{Logging in Multiprocessing}

When using multiprocessing, special care is needed to avoid conflicts:

\begin{lstlisting}
import logging
from logging.handlers import QueueHandler, QueueListener
from multiprocessing import Queue, Process
import time

def worker_process(queue, worker_id):
    """Worker process with queue-based logging"""
    # Configure worker to use queue
    qh = QueueHandler(queue)
    logger = logging.getLogger()
    logger.addHandler(qh)
    logger.setLevel(logging.INFO)
    
    # Do work with logging
    logger.info(f"Worker {worker_id} started")
    time.sleep(1)
    logger.info(f"Worker {worker_id} processing")
    time.sleep(1)
    logger.info(f"Worker {worker_id} completed")

if __name__ == '__main__':
    # Create queue for log records
    log_queue = Queue()
    
    # Setup handlers that will process records
    console_handler = logging.StreamHandler()
    file_handler = logging.FileHandler('multiprocess.log')
    
    formatter = logging.Formatter(
        '%(asctime)s - [PID:%(process)d] - %(levelname)s - %(message)s'
    )
    console_handler.setFormatter(formatter)
    file_handler.setFormatter(formatter)
    
    # Create listener to process queue
    listener = QueueListener(
        log_queue,
        console_handler,
        file_handler,
        respect_handler_level=True
    )
    listener.start()
    
    # Create and start worker processes
    processes = []
    for i in range(4):
        p = Process(target=worker_process, args=(log_queue, i))
        p.start()
        processes.append(p)
    
    # Wait for all workers
    for p in processes:
        p.join()
    
    # Stop listener
    listener.stop()
    
    print("All workers completed")
\end{lstlisting}

\subsection{Structured Logging with JSON}

For better log parsing and analysis, use JSON format:

\begin{lstlisting}
import logging
import json
from datetime import datetime

class JsonFormatter(logging.Formatter):
    """Format logs as JSON"""
    
    def format(self, record):
        log_data = {
            'timestamp': datetime.utcnow().isoformat(),
            'level': record.levelname,
            'logger': record.name,
            'message': record.getMessage(),
            'module': record.module,
            'function': record.funcName,
            'line': record.lineno
        }
        
        # Add exception info if present
        if record.exc_info:
            log_data['exception'] = self.formatException(record.exc_info)
        
        # Add extra fields
        if hasattr(record, 'user_id'):
            log_data['user_id'] = record.user_id
        if hasattr(record, 'request_id'):
            log_data['request_id'] = record.request_id
        
        return json.dumps(log_data)

# Usage
logger = logging.getLogger(__name__)
handler = logging.FileHandler('app.json.log')
handler.setFormatter(JsonFormatter())
logger.addHandler(handler)

logger.info("User logged in", extra={'user_id': 'user123'})
# Output: {"timestamp": "2024-12-17T14:30:45.123456", 
#          "level": "INFO", "logger": "__main__", 
#          "message": "User logged in", "user_id": "user123", ...}
\end{lstlisting}

\newpage

% ========================
% SECTION 10: TROUBLESHOOTING
% ========================
\section{Common Issues and Solutions}

\subsection{Duplicate Log Messages}

\textbf{Problem}: Log messages appear multiple times in output.

\vspace{0.5em}

\textbf{Cause}: Multiple handlers attached to logger or propagation causing duplicates.

\vspace{0.5em}
\textbf{Solutions}:

\begin{lstlisting}
# Solution 1: Check if handlers already exist
logger = logging.getLogger(__name__)
if not logger.handlers:
    handler = logging.StreamHandler()
    logger.addHandler(handler)

# Solution 2: Clear existing handlers
logger.handlers = []
logger.addHandler(new_handler)

# Solution 3: Disable propagation
logger.propagate = False

# Solution 4: Use hasHandlers() to check
if not logger.hasHandlers():
    logger.addHandler(handler)
\end{lstlisting}

\subsection{Logs Not Appearing}

\textbf{Problem}: Log messages don't appear in expected output.

\vspace{0.5em}
\textbf{Causes and Solutions}:

\begin{enumerate}[leftmargin=*]
    \item \textbf{Logger level too high}
    \begin{lstlisting}
# Check and set logger level
logger = logging.getLogger(__name__)
print(f"Current level: {logger.level}")
logger.setLevel(logging.DEBUG)
    \end{lstlisting}
    
    \item \textbf{Handler level too high}
    \begin{lstlisting}
# Set handler level appropriately
for handler in logger.handlers:
    print(f"Handler level: {handler.level}")
    handler.setLevel(logging.DEBUG)
    \end{lstlisting}
    
    \item \textbf{No handlers attached}
    \begin{lstlisting}
# Verify handlers exist
print(f"Handlers: {logger.handlers}")
if not logger.handlers:
    logger.addHandler(logging.StreamHandler())
    \end{lstlisting}
    
    \item \textbf{Root logger not configured}
    \begin{lstlisting}
# Configure root logger
logging.basicConfig(level=logging.DEBUG)
    \end{lstlisting}
\end{enumerate}

\subsection{basicConfig() Not Working}

\textbf{Problem}: \texttt{basicConfig()} appears to have no effect.

\vspace{0.5em}
\textbf{Cause}: \texttt{basicConfig()} only works once and only if root logger has no handlers.

\vspace{0.5em}
\textbf{Solutions}:

\begin{lstlisting}
# Solution 1: Use force=True (Python 3.8+)
logging.basicConfig(
    level=logging.DEBUG,
    format='%(asctime)s - %(levelname)s - %(message)s',
    force=True  # Reconfigure even if already configured
)

# Solution 2: Manually reset root logger
root = logging.getLogger()
for handler in root.handlers[:]:
    root.removeHandler(handler)
logging.basicConfig(level=logging.DEBUG)

# Solution 3: Check if already configured
if not logging.getLogger().hasHandlers():
    logging.basicConfig(level=logging.DEBUG)
\end{lstlisting}

\subsection{File Handler Not Writing}

\textbf{Problem}: Log file is not created or not being written to.

\vspace{0.5em}
\textbf{Solutions}:

\begin{enumerate}[leftmargin=*]
    \item \textbf{Check file permissions and path}
    \begin{lstlisting}
import os

log_file = 'app.log'
log_dir = os.path.dirname(log_file) or '.'

# Check if directory exists
if not os.path.exists(log_dir):
    os.makedirs(log_dir)

# Check write permissions
if not os.access(log_dir, os.W_OK):
    print(f"No write permission for {log_dir}")
    \end{lstlisting}
    
    \item \textbf{Ensure proper levels set}
    \begin{lstlisting}
logger.setLevel(logging.DEBUG)
file_handler.setLevel(logging.DEBUG)
    \end{lstlisting}
    
    \item \textbf{Force flush after logging}
    \begin{lstlisting}
for handler in logger.handlers:
    handler.flush()
    \end{lstlisting}
\end{enumerate}

\subsection{Performance Issues}

\textbf{Problem}: Logging causes performance degradation.

\vspace{0.5em}
\textbf{Solutions}:

\begin{enumerate}[leftmargin=*]
    \item \textbf{Use lazy formatting}
    \begin{lstlisting}
# Good - only formats if logged
logger.debug("Value: %s", expensive_computation())

# Bad - always computes and formats
logger.debug(f"Value: {expensive_computation()}")
    \end{lstlisting}
    
    \item \textbf{Increase logging level in production}
    \begin{lstlisting}
# Development
logger.setLevel(logging.DEBUG)

# Production
logger.setLevel(logging.INFO)
    \end{lstlisting}
    
    \item \textbf{Use QueueHandler for async logging}
    \begin{lstlisting}
from logging.handlers import QueueHandler
import queue

log_queue = queue.Queue()
queue_handler = QueueHandler(log_queue)
logger.addHandler(queue_handler)
    \end{lstlisting}
\end{enumerate}

\newpage

% ========================
% SECTION 11: QUICK REFERENCE
% ========================
\section{Quick Reference Guide}

\subsection{Essential Commands Cheat Sheet}

\begin{tcolorbox}[colback=green!5!white,colframe=green!75!black,title=Basic Setup]
\begin{lstlisting}
import logging

# Quick setup
logging.basicConfig(level=logging.INFO)

# Create logger
logger = logging.getLogger(__name__)

# Set level
logger.setLevel(logging.DEBUG)
\end{lstlisting}
\end{tcolorbox}

\begin{tcolorbox}[colback=blue!5!white,colframe=blue!75!black,title=Logging Messages]
\begin{lstlisting}
# Five standard levels
logger.debug("Detailed diagnostic information")
logger.info("Informational message")
logger.warning("Warning message")
logger.error("Error message")
logger.critical("Critical error message")

# Exception logging (includes traceback)
try:
    risky_operation()
except Exception:
    logger.exception("Operation failed")
\end{lstlisting}
\end{tcolorbox}

\begin{tcolorbox}[colback=purple!5!white,colframe=purple!75!black,title=Handlers and Formatters]
\begin{lstlisting}
# Create handlers
console_handler = logging.StreamHandler()
file_handler = logging.FileHandler('app.log')

# Rotating file handler
from logging.handlers import RotatingFileHandler
rotating_handler = RotatingFileHandler(
    'app.log',
    maxBytes=10*1024*1024,  # 10MB
    backupCount=5
)

# Create formatter
formatter = logging.Formatter(
    '%(asctime)s - %(name)s - %(levelname)s - %(message)s'
)

# Attach formatter to handler
handler.setFormatter(formatter)

# Add handler to logger
logger.addHandler(handler)
\end{lstlisting}
\end{tcolorbox}

\begin{tcolorbox}[colback=orange!5!white,colframe=orange!75!black,title=Complete Setup Example]
\begin{lstlisting}
import logging

logger = logging.getLogger(__name__)
logger.setLevel(logging.DEBUG)

# Console handler
console = logging.StreamHandler()
console.setLevel(logging.INFO)
console.setFormatter(logging.Formatter('%(levelname)s - %(message)s'))

# File handler
file_h = logging.FileHandler('app.log')
file_h.setLevel(logging.DEBUG)
file_h.setFormatter(logging.Formatter(
    '%(asctime)s - %(name)s - %(levelname)s - %(message)s'
))

logger.addHandler(console)
logger.addHandler(file_h)

logger.info("Logger configured")
\end{lstlisting}
\end{tcolorbox}

\subsection{Logging Levels Quick Reference}

\begin{center}
\begin{tabular}{|c|l|l|p{0.4\textwidth}|}
\hline
\textbf{Level} & \textbf{Numeric} & \textbf{Method} & \textbf{When to Use} \\
\hline
DEBUG & 10 & \texttt{logger.debug()} & Detailed diagnostic info \\
\hline
INFO & 20 & \texttt{logger.info()} & General informational messages \\
\hline
WARNING & 30 & \texttt{logger.warning()} & Warning messages \\
\hline
ERROR & 40 & \texttt{logger.error()} & Error messages \\
\hline
CRITICAL & 50 & \texttt{logger.critical()} & Critical errors \\
\hline
\end{tabular}
\end{center}

\subsection{Format String Placeholders}

\begin{center}
\begin{tabular}{|l|l|}
\hline
\textbf{Placeholder} & \textbf{Description} \\
\hline
\texttt{\%(name)s} & Logger name \\
\texttt{\%(levelname)s} & Log level name \\
\texttt{\%(message)s} & Log message \\
\texttt{\%(asctime)s} & Timestamp \\
\texttt{\%(filename)s} & Source filename \\
\texttt{\%(lineno)d} & Line number \\
\texttt{\%(funcName)s} & Function name \\
\texttt{\%(process)d} & Process ID \\
\texttt{\%(thread)d} & Thread ID \\
\hline
\end{tabular}
\end{center}

\subsection{Best Practices Checklist}

\begin{itemize}[leftmargin=*]
    \item[$\square$] Use \texttt{logging.getLogger(\_\_name\_\_)} for logger names
    \item[$\square$] Configure logging once at application startup
    \item[$\square$] Use appropriate log levels (DEBUG in dev, INFO+ in prod)
    \item[$\square$] Use lazy formatting: \texttt{logger.info("User \%s", user)}
    \item[$\square$] Use \texttt{logger.exception()} in except blocks
    \item[$\square$] Never log sensitive information (passwords, keys)
    \item[$\square$] Implement log rotation for production systems
    \item[$\square$] Add timestamps to all log messages
    \item[$\square$] Include context (request ID, user ID) when relevant
    \item[$\square$] Test logging configuration before deploying
    \item[$\square$] Monitor log file sizes regularly
    \item[$\square$] Use structured logging (JSON) for complex systems
    \item[$\square$] Set up centralized logging for distributed systems
    \item[$\square$] Document logging conventions for your team
\end{itemize}

\newpage

% ========================
% SECTION 12: GLOSSARY
% ========================
\section{Glossary}

\begin{description}[leftmargin=4cm,style=nextline]
    \item[Logger] The main interface that applications use to log messages. Loggers are organized hierarchically by name.
    
    \item[Handler] Object responsible for dispatching log records to specific destinations (console, file, network, etc.).
    
    \item[Formatter] Specifies the layout of log messages, including which attributes to include and how to format them.
    
    \item[Filter] Provides fine-grained control over which log records are processed, beyond simple level filtering.
    
    \item[Log Level] Severity indicator for log messages (DEBUG=10, INFO=20, WARNING=30, ERROR=40, CRITICAL=50).
    
    \item[Log Record] Object containing all information about a single logging event (message, level, timestamp, location, etc.).
    
    \item[Propagation] Process by which log messages are passed from child loggers to parent loggers in the hierarchy.
    
    \item[Root Logger] The top-level logger in the hierarchy (\texttt{logging.getLogger()} with no name), parent to all other loggers.
    
    \item[StreamHandler] Handler that sends log output to streams like sys.stdout or sys.stderr (console output).
    
    \item[FileHandler] Handler that writes log messages to a file on disk.
    
    \item[RotatingFileHandler] Handler that automatically rotates log files when they reach a specified size.
    
    \item[TimedRotatingFileHandler] Handler that rotates log files at specified time intervals (hourly, daily, etc.).
    
    \item[basicConfig()] Convenience function for simple logging configuration, works only once per program.
    
    \item[Lazy Formatting] Technique where string formatting is delayed until the message is actually logged, improving performance.
    
    \item[Structured Logging] Practice of logging data in a structured format (e.g., JSON) rather than plain text, making logs easier to parse and analyze.
    
    \item[QueueHandler] Handler that sends log records to a queue for asynchronous processing, useful in multiprocessing environments.
    
    \item[QueueListener] Receives log records from a queue and dispatches them to configured handlers.
    
    \item[Logger Hierarchy] Tree structure of loggers where child loggers inherit configuration from parent loggers.
    
    \item[Exception Logging] Special logging that includes full traceback information, typically done with \texttt{logger.exception()}.
    
    \item[Handler Level] Minimum severity level a handler will output, independent of the logger's level.
    
    \item[Logger Level] Minimum severity level a logger will process before passing to handlers.
    
    \item[Context Information] Additional data attached to log records (user ID, request ID, session info) for better traceability.
\end{description}

\newpage

% ========================
% CONCLUSION
% ========================
\section{Conclusion}

Effective logging is a cornerstone of professional software development and absolutely essential for Machine Learning Operations (MLOps). The Python logging module provides a robust, flexible, and production-ready framework for tracking application behavior, debugging issues, and monitoring systems in production environments.

\subsection{Key Takeaways}

\begin{enumerate}[leftmargin=*]
    \item \textbf{Never Use print() in Production}
    
    Always use the logging module instead of print() statements. Logging provides:
    \begin{itemize}
        \item Severity levels for filtering
        \item Flexible output destinations
        \item Automatic timestamps and context
        \item Production-ready features
    \end{itemize}
    
    \item \textbf{Choose Appropriate Log Levels}
    
    Use the right level for each situation:
    \begin{itemize}
        \item DEBUG for development and detailed diagnostics
        \item INFO for production tracking and milestones
        \item WARNING for potential issues that don't stop execution
        \item ERROR for failures that affect functionality
        \item CRITICAL for system-level failures
    \end{itemize}
    
    \item \textbf{Configure Logging Properly}
    
    Set up logging at application startup with:
    \begin{itemize}
        \item Appropriate handlers (console, file, rotating)
        \item Clear, informative formatters
        \item Correct levels for each environment
        \item Log rotation to manage disk space
    \end{itemize}
    
    \item \textbf{Log at the Right Verbosity}
    
    Balance information and noise:
    \begin{itemize}
        \item Too much logging creates noise and performance issues
        \item Too little logging misses important events
        \item Use DEBUG liberally in development
        \item Be selective with INFO/WARNING in production
    \end{itemize}
    
    \item \textbf{Include Relevant Context}
    
    Make logs actionable by including:
    \begin{itemize}
        \item Request IDs for tracing requests
        \item User IDs for user-specific issues
        \item Timestamps for temporal analysis
        \item Function names and line numbers for debugging
    \end{itemize}
    
    \item \textbf{Handle Errors Gracefully}
    
    Always use proper exception logging:
    \begin{itemize}
        \item Use \texttt{logger.exception()} in except blocks
        \item Include full tracebacks for debugging
        \item Log before re-raising exceptions
        \item Don't swallow exceptions silently
    \end{itemize}
    
    \item \textbf{Monitor and Rotate Logs}
    
    In production systems:
    \begin{itemize}
        \item Implement log rotation to prevent disk full
        \item Monitor log file sizes regularly
        \item Archive old logs appropriately
        \item Set up alerts for critical errors
    \end{itemize}
\end{enumerate}

\subsection{MLOps-Specific Recommendations}

For machine learning operations, logging serves critical functions:

\begin{itemize}[leftmargin=*]
    \item \textbf{Model Training}: Track epochs, loss, metrics, and training time
    \item \textbf{Data Pipelines}: Log data loading, cleaning, and transformation steps
    \item \textbf{Model Serving}: Record prediction requests, responses, and inference times
    \item \textbf{Performance Monitoring}: Track resource usage (CPU, GPU, memory)
    \item \textbf{Error Tracking}: Identify and debug production failures quickly
    \item \textbf{Audit Trails}: Maintain compliance and reproducibility records
    \item \textbf{Experiment Tracking}: Document configurations and results
\end{itemize}

\subsection{Implementation Strategy}

When implementing logging in your projects:

\begin{enumerate}[leftmargin=*]
    \item \textbf{Start Simple}: Begin with \texttt{basicConfig()} for prototypes
    \item \textbf{Expand Gradually}: Add handlers and formatters as needs grow
    \item \textbf{Establish Standards}: Define logging conventions for your team
    \item \textbf{Review Regularly}: Analyze logs to improve logging strategy
    \item \textbf{Integrate Tools}: Connect logging with monitoring and alerting systems
    \item \textbf{Test Configuration}: Verify logging works before deploying
    \item \textbf{Document Practices}: Maintain clear documentation for team members
\end{enumerate}

\subsection{Common Pitfalls to Avoid}

\begin{warningbox}
\textbf{Avoid These Common Mistakes}:
\begin{itemize}[leftmargin=*]
    \item Using print() instead of logging in production
    \item Logging sensitive information (passwords, API keys)
    \item Over-logging in tight loops (performance impact)
    \item Not rotating log files (disk space issues)
    \item Using string concatenation instead of lazy formatting
    \item Forgetting to set appropriate log levels
    \item Not including enough context in log messages
    \item Swallowing exceptions without logging
    \item Configuring logging multiple times
    \item Not testing logging configuration
\end{itemize}
\end{warningbox}

\subsection{Next Steps}

To continue improving your logging practices:

\begin{enumerate}[leftmargin=*]
    \item \textbf{Practice}: Implement logging in personal projects
    \item \textbf{Experiment}: Try different handlers and formatters
    \item \textbf{Integrate}: Connect logging with monitoring tools (ELK, Splunk, Datadog)
    \item \textbf{Learn Advanced Topics}: Explore async logging, custom handlers, filters
    \item \textbf{Study Production Systems}: Analyze logging in open-source projects
    \item \textbf{Optimize}: Profile and optimize logging performance
    \item \textbf{Automate}: Set up automated log analysis and alerting
\end{enumerate}

\subsection{Integration with Other Tools}

Python logging integrates well with:

\begin{itemize}[leftmargin=*]
    \item \textbf{MLflow}: Experiment tracking and model registry
    \item \textbf{TensorBoard}: Visualization and monitoring
    \item \textbf{ELK Stack}: Elasticsearch, Logstash, Kibana for log analysis
    \item \textbf{Splunk}: Enterprise log management
    \item \textbf{Datadog}: Cloud monitoring and analytics
    \item \textbf{Sentry}: Error tracking and monitoring
    \item \textbf{CloudWatch}: AWS log management
    \item \textbf{Prometheus/Grafana}: Metrics and dashboards
\end{itemize}

\subsection{Final Recommendations}

\begin{infobox}{Best Practices Summary}
\begin{enumerate}[leftmargin=*]
    \item Use \texttt{logging.getLogger(\_\_name\_\_)} for all loggers
    \item Configure logging once at application startup
    \item Use appropriate levels: DEBUG $<$ INFO $<$ WARNING $<$ ERROR $<$ CRITICAL
    \item Implement log rotation in production
    \item Include context information (timestamps, IDs, locations)
    \item Use lazy formatting for performance
    \item Never log sensitive information
    \item Use \texttt{logger.exception()} in except blocks
    \item Test logging configuration thoroughly
    \item Monitor and analyze logs regularly
\end{enumerate}
\end{infobox}

\subsection{Additional Resources}

For further learning and reference:

\begin{itemize}[leftmargin=*]
    \item \textbf{Official Documentation}: 
    \begin{itemize}
        \item Python Logging Module: \url{https://docs.python.org/3/library/logging.html}
        \item Logging Cookbook: \url{https://docs.python.org/3/howto/logging-cookbook.html}
        \item Logging HOWTO: \url{https://docs.python.org/3/howto/logging.html}
    \end{itemize}
    
    \item \textbf{Tutorials and Guides}:
    \begin{itemize}
        \item Real Python Logging Guide: \url{https://realpython.com/python-logging/}
        \item Python Logging Best Practices: \url{https://www.loggly.com/ultimate-guide/python-logging-basics/}
    \end{itemize}
    
    \item \textbf{MLOps Integration}:
    \begin{itemize}
        \item MLflow Documentation: \url{https://www.mlflow.org/docs/latest/tracking.html}
        \item TensorBoard Logging: \url{https://www.tensorflow.org/tensorboard}
    \end{itemize}
    
    \item \textbf{Log Management Tools}:
    \begin{itemize}
        \item ELK Stack: \url{https://www.elastic.co/elastic-stack}
        \item Splunk: \url{https://www.splunk.com/}
        \item Datadog: \url{https://www.datadoghq.com/}
        \item Sentry: \url{https://sentry.io/}
    \end{itemize}
    
    \item \textbf{Books}:
    \begin{itemize}
        \item "Logging and Log Management" by Anton Chuvakin
        \item "Python Logging Essentials" by Chetan Giridhar
    \end{itemize}
\end{itemize}

\vspace{2em}

\begin{tcolorbox}[colback=blue!5!white,colframe=blue!75!black,title=Remember]
\textbf{Logging is not just about recording events—it's about making your application observable, debuggable, and maintainable.}

Good logging practices will:
\begin{itemize}[leftmargin=*]
    \item Save hours of debugging time
    \item Enable quick problem resolution in production
    \item Provide insights into system behavior
    \item Build confidence in your deployments
    \item Support compliance and auditing requirements
    \item Facilitate team collaboration
\end{itemize}

Invest time in proper logging setup—it pays dividends throughout the application lifecycle!
\end{tcolorbox}

\vspace{2em}
\hrule
\vspace{0.5em}
\begin{center}
\textit{End of Python Logging Module Complete Reference Guide}

\vspace{1em}

\textit{"The most effective debugging tool is still careful thought,\\
coupled with judiciously placed print statements."}\\
\textit{— Brian Kernighan}

\vspace{0.5em}

\end{center}

\end{document}

